\documentclass{beamer}

\usepackage{cmbright} % Математика "без засечек"
\usepackage{fontspec}
\setsansfont{CMU Sans Serif}
\usepackage[russian]{babel}
\usepackage{pgfplots} % Диаграмма
\hypersetup{unicode=true} % Кириллица в закладках PDF

\begin{document}

Освящение Геттисбергского кладбища

Авраам Линкольн

Соединённые Штаты Америки

19 ноября 1863 г.

\section{Повестка дня}

Повестка дня

  * Встретились на поле битвы (великом)
  * Освятить часть поля --- уместно!
  * Неоконченное дело (великие задачи)

Невозможно включить в повестку дня!

  * Освятить
  * Сделать священной
  * Одухотворить (в узком смысле)
  * Прибавить или убавить
  * Заметить или запомнить, что было сказано

\section{Обзор}

Ключевые задачи \& составляющие успеха

Что делает нацию уникальной
  * Рождённая в свободе
  * Все люди равны

Общее видение:
  * Возрождение свободы.
  * Власть народа, волей народа и для народа.

Организационный обзор

\begin{tikzpicture}
\begin{axis}[
  width=7cm,
  height=5cm,
  symbolic x coords={$-87$ лет, сейчас},
  xtick=data,
  nodes near coords,
  ybar=6pt,
  enlargelimits=0.2,
  every axis plot/.append style=thick,
  axis line style = thick
  ]
\addplot coordinates {
  ($-87$ лет,   1)
  (сейчас,  0)
};
\legend{Новые нации};
\end{axis}
\end{tikzpicture}

Четыре двадцатилетия и семь

-(4 * 20 + 7) = -87

\section{Резюме}

* Новая нация
* Гражданская война
* Освятить поле

* Посвятим себя неоконченному делу
* Возрождение свободы
* Нация не исчезнет с лица земли

\end{document}
