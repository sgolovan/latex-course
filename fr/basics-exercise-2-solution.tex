\documentclass{article}
\usepackage[utf8]{inputenc} % obligatoire pour Overleaf
\usepackage{amsmath}
\begin{document}

Soit $X_1, X_2, \ldots, X_n$ une suite de variables aléatoires
indépendantes et identiquement distribuées avec
$\operatorname{E}[X_i] = \mu$ et $\operatorname{Var}[X_i] = \sigma^2 < \infty$,
et soit
\begin{equation*}
S_n = \frac{1}{n}\sum_{i=1}^{n} X_i
\end{equation*}
leur moyenne. Alors quand $n$ tend vers l'infini,
la racine carrée des variables aléatoires $\sqrt{n}(S_n - \mu)$
converge en distribution vers la loi normale $N(0, \sigma^2)$.

% points bonus: le N de la loi normale est normalement composé en police
% calligraphique ; vous pouvez l'obtenir en tapant $\mathcal{N}(0, \sigma^2)$.

\end{document}
