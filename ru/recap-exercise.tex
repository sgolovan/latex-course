\documentclass[12pt]{article}

\usepackage{fontspec}
\setmainfont{CMU Serif}
\usepackage[russian]{babel}
\usepackage{url}

\begin{document}
Десять секретов хорошего научного доклада

Автор: Вы

Введение

Текст для этого упражнения - существенно сокращённая и слегка модифицированная версия прекрасной статьи с таким же названием за
авторством Марка Шёберла и Брайана Туна:
\url{http://www.cgd.ucar.edu/cms/agu/scientific_talk.html}

Секреты

Я собрал этот персональный список "секретов", слушая эффективных и неэффективных докладчиков. Я не претендую на то, что этот
список исчерпывающий - я уверен, что есть вещи, о которых я забыл упомянуть. Но мой список покрывает примерно 90% того, что вам
необходимо знать и делать.

1) Готовьте материал внимательно и логично. Рассказывайте историю.

2) Репетируйте ваш доклад. Не может быть никаких извинений за недостаточную подготовленность.

3) Не старайтесь запихнуть в доклад слишком много материала. Хороший докладчик имеет одну-две центральные темы и придерживается
их.

4) Избегайте формул. Говорят, что каждое уравнение в докладе приводит к тому, что число слушателей, понимающих его, уменьшается
наполовину. То есть, если мы обозначим через q число уравнений в вашем докладе, а через n - число слушателей, которые поймут ваш
доклад, то справедливо равенство

n = gamma (1/2) в степени q

где gamma - коэффициент пропорциональности.

5) Ограничьтесь только несколькими выводами. Слушатели не могут помнить более, чем пару вещей из доклада, особенно если они
слушают много докладов на большой конференции.

6) Разговаривайте с аудиторией, не с экраном. Одна из наиболее часто встречающихся проблем, которые я встречал, заключается в
том, что докладчик говорит, отвернувшись к экрану.

7) Старайтесь не издавать отвлекающие звуки. Избегайте произносить "Эммм" или "Ээээ" между предложениями.

8) Совершенствуйте графику. Вот список правил, которые помогут сделать графику лучше:

* Используйте крупные буквы.

* Не переусложняйте рисунки. Не показывайте графики, которые вам не понадобятся.

* Применяйте выделение цветом.

9) Не огрызайтесь, когда вам задают вопросы.

10) Не чурайтесь юмора. Я всегда поражаюсь, как даже самые глупые шутки вызывают смех во время научного доклада.
\end{document}
