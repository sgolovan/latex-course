\documentclass[12pt]{article}
\usepackage{amsmath}
\usepackage{fontspec}
\setmainfont{CMU Serif}
\usepackage[russian]{babel}
\usepackage{url}

\title{Десять секретов хорошего научного доклада}
\author{Вы}

\begin{document}
\maketitle

\section{Введение}

Текст для этого упражнения --- существенно сокращённая и слегка модифицированная версия прекрасной статьи с таким же названием за
авторством Марка Шёберла и Брайана Туна:
\url{http://www.cgd.ucar.edu/cms/agu/scientific_talk.html}

\section{Секреты}

Я собрал этот персональный список <<секретов>>, слушая эффективных и неэффективных докладчиков. Я не претендую на то, что этот
список исчерпывающий --- я уверен, что есть вещи, о которых я забыл упомянуть. Но мой список покрывает примерно 90\% того, что вам
необходимо знать и делать.

\begin{enumerate}
\item Готовьте материал внимательно и логично. Рассказывайте историю.
\item Репетируйте ваш доклад. Не может быть никаких извинений за недостаточную подготовленность.
\item Не старайтесь запихнуть в доклад слишком много материала. Хороший докладчик имеет одну-две центральные темы и придерживается
их.
\item Избегайте формул. Говорят, что каждое уравнение в докладе приводит к тому, что число слушателей, понимающих его, уменьшается
наполовину. То есть, если мы обозначим через $q$ число уравнений в вашем докладе, а через $n$ --- число слушателей, которые поймут ваш
доклад, то справедливо равенство
\begin{equation*}
  n = \gamma\left(\frac{1}{2}\right)^q,
\end{equation*}
где $\gamma$ --- коэффициент пропорциональности.
\item Ограничьтесь только несколькими выводами. Слушатели не могут помнить более, чем пару вещей из доклада, особенно если они
слушают много докладов на большой конференции.
\item Разговаривайте с аудиторией, не с экраном. Одна из наиболее часто встречающихся проблем, которые я встречал, заключается в
том, что докладчик говорит, отвернувшись к экрану.
\item Старайтесь не издавать отвлекающие звуки. Избегайте произносить <<Эммм>> или <<Ээээ>> между предложениями.
\item Совершенствуйте графику. Вот список правил, которые помогут сделать графику лучше:
\begin{itemize}
  \item Используйте крупные буквы.
  \item Не переусложняйте рисунки. Не показывайте графики, которые вам не понадобятся.
  \item Применяйте выделение цветом.
\end{itemize}
\item Не огрызайтесь, когда вам задают вопросы.
\item Не чурайтесь юмора. Я всегда поражаюсь, как даже самые глупые шутки вызывают смех во время научного доклада.
\end{enumerate}
\end{document}
