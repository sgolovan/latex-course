\documentclass{beamer}

%
% Common preamble for all three parts.
%

\usetheme[block=fill]{metropolis}
\setbeamercolor{frametitle}{use=normal text, parent=normal text}

\usepackage{arevmath}
\SetSymbolFont{largesymbols}{normal}{OMX}{iwona}{m}{n}
\usepackage{fontspec}
\setmainfont{PT Sans}
\setsansfont{PT Sans}
\setmonofont{PT Mono}[Scale=0.87]
\usepackage[english,russian]{babel}
\usepackage{amsmath}
\usepackage{color}
\usepackage{minted}
\usepackage{hyperref}
\usepackage{multicol}
\usepackage{tabularx}
\usepackage{tikz}
\usepackage{tcolorbox}
\usepackage{xparse}

% For slide 28 (Tikz examples)
\usetikzlibrary{mindmap,trees}
\usetikzlibrary{backgrounds,shapes,arrows,positioning,calc,snakes,fit}
\usepgflibrary{decorations.markings}

\hypersetup{unicode=true}

\tcbuselibrary{skins}
\tcbuselibrary{listings}
\tcbuselibrary{minted}
\tcbset{colframe=mDarkTeal, colback=white!90!mDarkTeal,% Taken from the Metropolis theme
        left=0.8em,right=0.8em}
\newtcolorbox{tblock}[1]{boxsep=1mm,sidebyside=false,bicolor=false,colback=white,title={#1}}
\newtcolorbox{printout}{boxsep=0mm,sidebyside=false,bicolor=false,colback=white}
\def\linkbox#1{\tcbox[on line,boxsep=0mm,left=2pt,right=2pt,top=2pt,bottom=2pt,
                      colback=mDarkTeal,coltext=white]{#1}}
\def\ovllink#1{\tcbox[on line,boxsep=0mm,left=2pt,right=2pt,top=2pt,bottom=2pt,
                      colback=green!30!black,colframe=green!30!black,coltext=white]{#1}}
\newtcblisting{code}{boxsep=0mm,listing only,minted language=latex}
\newtcblisting{bibtexcode}{boxsep=0mm,listing only,minted language=bibtex}
\newtcblisting{exampletwoup}{fontupper=\small,fontlower=\small,
                             boxsep=0mm,listing side text,minted language=latex,
                             bicolor,colbacklower=white,
                             righthand ratio=0.42}
\newtcblisting{exampletwouptiny}{fontupper=\footnotesize,fontlower=\footnotesize,
                                 boxsep=0mm,listing side text,minted language=latex,
                                 minted options={fontsize=\footnotesize},
                                 bicolor,colbacklower=white,
                                 righthand ratio=0.42}
\newtcblisting{exampletwouppaused}{fontupper=\footnotesize,fontlower=\footnotesize,
                             boxsep=0mm,listing side text,minted language=latex,
                             minted options={fontsize=\footnotesize},
                             bicolor,colbacklower=white,
                             righthand ratio=0.42,after lower=\onslide<1->}
\newtcbinputlisting{\inputcode}[2][]{%
listing file={#2},boxsep=0mm,listing only,minted language=latex,#1}
\newtcbinputlisting{\inputbibtexcode}[2][]{%
listing file={#2},boxsep=0mm,listing only,minted language=bibtex,#1}

\DeclareDocumentCommand{\linkbox}{O{mDarkTeal}m}{%
\tcbox[on line,
       boxsep=0mm,
       left=2pt,
       right=2pt,
       top=2pt,
       bottom=2pt,
       colframe=#1,
       colback=#1,
       coltext=white]{#2}}

\newcommand{\genhref}[2]{\href{#1}{\linkbox{#2}}}
\newcommand{\filhref}[2]{\href{\fileuri/#1}{\linkbox{#2}}}
\newcommand{\ovlhref}[2]{\href{\wlnewdoc{#1}}{\linkbox[green!30!black]{#2}}}

% only inline todonotes work
\usepackage{xkeyval}
\usepackage[textsize=small]{todonotes}
\presetkeys{todonotes}{inline}{}

\usetikzlibrary{shapes,arrows,positioning,shadows}

% no nav buttons
\usenavigationsymbolstemplate{}

\newcommand{\bftt}[1]{\textbf{\texttt{#1}}}
%\newcommand{\comment}[1]{{\color[HTML]{008080}\textit{\textbf{\texttt{#1}}}}}
\newcommand{\cmd}[1]{{\color[HTML]{008000}\bftt{#1}}}
\newcommand{\bs}{\char`\\}
\newcommand{\cmdbs}[1]{\cmd{\bs#1}}
\newcommand{\lcb}{\char '173}
\newcommand{\rcb}{\char '175}
\newcommand{\cmdbegin}[1]{\cmdbs{begin\lcb}\bftt{#1}\cmd{\rcb}}
\newcommand{\cmdend}[1]{\cmdbs{end\lcb}\bftt{#1}\cmd{\rcb}}

\newcommand{\wllogo}{\textbf{Overleaf}}

% this is where the example source files are loaded from
% do not include a trailing slash
\newcommand{\fileuri}{https://raw.github.com/sgolovan/latex-course/master/ru}

\newcommand{\wlserver}{https://www.overleaf.com}
\newcommand{\wlnewdoc}[1]{\wlserver/docs?snip\_uri=\fileuri/#1\&splash=none}

\def\tikzname{Ti\emph{k}Z}

% from http://tex.stackexchange.com/questions/5226/keyboard-font-for-latex
\newcommand*\keystroke[1]{%
  \tikz[baseline=(key.base)]
    \node[%
      draw,
      fill=white,
      drop shadow={shadow xshift=0.25ex,shadow yshift=-0.25ex,fill=black,opacity=0.75},
      rectangle,
      rounded corners=2pt,
      inner sep=1pt,
      line width=0.5pt,
      font=\scriptsize\ttfamily
    ](key) {#1\strut}
  ;
}
\newcommand{\keystrokebftt}[1]{\keystroke{\bftt{#1}}}

% stolen from minted.dtx
\newenvironment{exampletwouptinynoframe}
  {\VerbatimEnvironment
   \begin{VerbatimOut}{example.out}}
  {\end{VerbatimOut}
   \setlength{\parindent}{0pt}
   \begin{tabular}{l|l}
   \begin{minipage}{0.55\linewidth}
     \inputminted[fontsize=\scriptsize,resetmargins]{latex}{example.out}
   \end{minipage} &
   \begin{minipage}{0.35\linewidth}
     \setlength{\parskip}{6pt plus 1pt minus 1pt}%
     \raggedright\scriptsize\input{example.out}
   \end{minipage}
   \end{tabular}}

\title{Интерактивное введение в \LaTeX}
\author{Джон Д. Лис-Миллер\\Перевод на русский язык: Сергей Головань}
%\titlegraphic{%
%\includegraphics[height=36pt]{overleaf}\\[1em]
%\includegraphics[height=24pt]{UoB-logo}
%}


\subtitle{Часть 2: Структурированные документы и не только}

\begin{document}

%%%%%%%%%%%%%%%%%%%%%%%%%%%%%%%%%%%%%%%%%%%%%%%%%%%%%%%%%%%%%%%%%%%%%%%%%%%%%%%
%%%%%%%%%%%%%%%%%%%%%%%%%%%%%%%%%%%%%%%%%%%%%%%%%%%%%%%%%%%%%%%%%%%%%%%%%%%%%%%
%%%%%%%%%%%%%%%%%%%%%%%%%%%%%%%%%%%%%%%%%%%%%%%%%%%%%%%%%%%%%%%%%%%%%%%%%%%%%%%
\begin{frame}
\titlepage
\end{frame}

%%%%%%%%%%%%%%%%%%%%%%%%%%%%%%%%%%%%%%%%%%%%%%%%%%%%%%%%%%%%%%%%%%%%%%%%%%%%%%%
%%%%%%%%%%%%%%%%%%%%%%%%%%%%%%%%%%%%%%%%%%%%%%%%%%%%%%%%%%%%%%%%%%%%%%%%%%%%%%%
%%%%%%%%%%%%%%%%%%%%%%%%%%%%%%%%%%%%%%%%%%%%%%%%%%%%%%%%%%%%%%%%%%%%%%%%%%%%%%%
\section{Структурированные документы}

%%%%%%%%%%%%%%%%%%%%%%%%%%%%%%%%%%%%%%%%%%%%%%%%%%%%%%%%%%%%%%%%%%%%%%%%%%%%%%%
%%%%%%%%%%%%%%%%%%%%%%%%%%%%%%%%%%%%%%%%%%%%%%%%%%%%%%%%%%%%%%%%%%%%%%%%%%%%%%%
%%%%%%%%%%%%%%%%%%%%%%%%%%%%%%%%%%%%%%%%%%%%%%%%%%%%%%%%%%%%%%%%%%%%%%%%%%%%%%%
\begin{frame}
\frametitle{План}
\vspace{-3ex}
\begin{multicols}{2}
\tableofcontents[currentsection]
\end{multicols}
\end{frame}

%%%%%%%%%%%%%%%%%%%%%%%%%%%%%%%%%%%%%%%%%%%%%%%%%%%%%%%%%%%%%%%%%%%%%%%%%%%%%%%
%%%%%%%%%%%%%%%%%%%%%%%%%%%%%%%%%%%%%%%%%%%%%%%%%%%%%%%%%%%%%%%%%%%%%%%%%%%%%%%
%%%%%%%%%%%%%%%%%%%%%%%%%%%%%%%%%%%%%%%%%%%%%%%%%%%%%%%%%%%%%%%%%%%%%%%%%%%%%%%
\begin{frame}
\frametitle{\insertsection}
\begin{itemize}
\item В первой части мы узнали о командах и об окружениях, использующихся
  для набора текста и математических формул.
\item Теперь мы изучим команды и окружения для структурирования документов.
\item Как и раньше, пробуйте вводить новые команды в \wllogo{}:
\end{itemize}
\vspace{2ex}
\begin{center}
\ovlhref{basics.tex}{Щёлкните здесь, чтобы открыть пример документа в \wllogo{}}

\vspace{1ex}
\scriptsize
Для лучшего результата воспользуйтесь браузером
\href{http://www.google.com/chrome}{Google Chrome} или новой версией браузера
\href{http://www.mozilla.org/en-US/firefox/new/}{FireFox}.
\end{center}
\vspace{1ex}
\begin{itemize}
\item Начнём же!
\end{itemize}
\end{frame}

%%%%%%%%%%%%%%%%%%%%%%%%%%%%%%%%%%%%%%%%%%%%%%%%%%%%%%%%%%%%%%%%%%%%%%%%%%%%%%%
%%%%%%%%%%%%%%%%%%%%%%%%%%%%%%%%%%%%%%%%%%%%%%%%%%%%%%%%%%%%%%%%%%%%%%%%%%%%%%%
%%%%%%%%%%%%%%%%%%%%%%%%%%%%%%%%%%%%%%%%%%%%%%%%%%%%%%%%%%%%%%%%%%%%%%%%%%%%%%%
\subsection{Язык и переносы}

\begin{frame}[fragile]
\frametitle{\insertsubsection}
\vspace{-2ex}
\begin{itemize}
\item Подключите шрифт, в котором есть русские буквы, с помощью пакета
  \bftt{fontspec}.
\item Укажите в необязательном аргументе при подключении пакета \bftt{babel},
  что ваш документ на русском языке.
\item Компилируйте ваш документ с помощью \bftt{lualatex}.
\begin{code}
\usepackage{fontspec}
\setmainfont{CMU Serif} % шрифт с русскими буквами
\usepackage[russian]{babel} % переносы
\end{code}
\item Вместо пакета \bftt{fontspec} можно подключить пакеты \bftt{fontenc}
  и \bftt{inputenc} и компилировать документ с помощью \bftt{pdflatex}.
\begin{code}
\usepackage[T2A]{fontenc} % шрифт с русскими буквами
\usepackage[utf8]{inputenc} % кодировка документа
\usepackage[russian]{babel} % переносы
\end{code}
\end{itemize}
\end{frame}

%%%%%%%%%%%%%%%%%%%%%%%%%%%%%%%%%%%%%%%%%%%%%%%%%%%%%%%%%%%%%%%%%%%%%%%%%%%%%%%
%%%%%%%%%%%%%%%%%%%%%%%%%%%%%%%%%%%%%%%%%%%%%%%%%%%%%%%%%%%%%%%%%%%%%%%%%%%%%%%
%%%%%%%%%%%%%%%%%%%%%%%%%%%%%%%%%%%%%%%%%%%%%%%%%%%%%%%%%%%%%%%%%%%%%%%%%%%%%%%
\subsection{Название и аннотация}

\begin{frame}[fragile]
\frametitle{\insertsubsection}
\vspace{-2ex}
\begin{itemize}{\small
\item Укажите \LaTeX{}'у \cmdbs{title} и \cmdbs{author} в преамбуле.
\item Далее воспользуйтесь \cmdbs{maketitle} в документе, чтобы вывести название.
\item Для печати аннотации применяется окружение \bftt{abstract}.
}\end{itemize}
\vspace{-2ex}
\begin{minipage}{0.55\linewidth}
\inputcode[minted options={fontsize=\footnotesize}]{structure-title.tex}
\end{minipage}~~%
\begin{minipage}{0.45\linewidth}
\includegraphics[width=\textwidth,clip,viewport=0cm 227mm 5cm 297mm]{structure-title.pdf}
\end{minipage}
\end{frame}

%%%%%%%%%%%%%%%%%%%%%%%%%%%%%%%%%%%%%%%%%%%%%%%%%%%%%%%%%%%%%%%%%%%%%%%%%%%%%%%
%%%%%%%%%%%%%%%%%%%%%%%%%%%%%%%%%%%%%%%%%%%%%%%%%%%%%%%%%%%%%%%%%%%%%%%%%%%%%%%
%%%%%%%%%%%%%%%%%%%%%%%%%%%%%%%%%%%%%%%%%%%%%%%%%%%%%%%%%%%%%%%%%%%%%%%%%%%%%%%
\subsection{Разделы}

\begin{frame}[fragile]
\frametitle{\insertsubsection}
\vspace{-2ex}
\begin{itemize}{\small
\item Просто указывайте \cmdbs{section} или \cmdbs{subsection}.
\item Угадайте, что делают команды \cmdbs{section*} и \cmdbs{subsection*}?
}\end{itemize}
\begin{minipage}{0.55\linewidth}
\inputcode[minted options={fontsize=\footnotesize}]{structure-sections.tex}
\end{minipage}~~%
\begin{minipage}{0.45\linewidth}
\includegraphics[width=\textwidth,clip,viewport=0cm 227mm 5cm 297mm]{structure-sections.pdf}
\end{minipage}
\end{frame}

%%%%%%%%%%%%%%%%%%%%%%%%%%%%%%%%%%%%%%%%%%%%%%%%%%%%%%%%%%%%%%%%%%%%%%%%%%%%%%%
%%%%%%%%%%%%%%%%%%%%%%%%%%%%%%%%%%%%%%%%%%%%%%%%%%%%%%%%%%%%%%%%%%%%%%%%%%%%%%%
%%%%%%%%%%%%%%%%%%%%%%%%%%%%%%%%%%%%%%%%%%%%%%%%%%%%%%%%%%%%%%%%%%%%%%%%%%%%%%%
\subsection{Метки и перекрёстные ссылки}

\begin{frame}[fragile]
\frametitle{\insertsubsection}
\vspace{-2ex}
\begin{itemize}{\small
\item Применяйте \cmdbs{label} и \cmdbs{ref} для автоматических ссылок.
\item Пакет \bftt{amsmath} предоставляет \cmdbs{eqref} для ссылок на уравнения.
}\end{itemize}
\begin{minipage}{0.55\linewidth}
\inputcode[minted options={fontsize=\scriptsize}]{structure-crossref.tex}
\end{minipage}~~%
\begin{minipage}{0.45\linewidth}
\includegraphics[width=\textwidth,clip,viewport=0cm 227mm 5cm 297mm]{structure-crossref.pdf}
\end{minipage}
\end{frame}

%%%%%%%%%%%%%%%%%%%%%%%%%%%%%%%%%%%%%%%%%%%%%%%%%%%%%%%%%%%%%%%%%%%%%%%%%%%%%%%
%%%%%%%%%%%%%%%%%%%%%%%%%%%%%%%%%%%%%%%%%%%%%%%%%%%%%%%%%%%%%%%%%%%%%%%%%%%%%%%
%%%%%%%%%%%%%%%%%%%%%%%%%%%%%%%%%%%%%%%%%%%%%%%%%%%%%%%%%%%%%%%%%%%%%%%%%%%%%%%
\subsection{Упражнение}

\begin{frame}[fragile]
\frametitle{Упражнение на структурные документы}
\begin{tblock}{Наберите этот короткий документ в \LaTeX:\footnotemark}
\begin{center}
\filhref{structure-exercise-solution.pdf}{Щёлкните, чтобы открыть документ для сравнения}
\end{center}
Попытайтесь добиться того, чтобы ваша статья выглядела как эта. Используйте
\cmdbs{ref} и \cmdbs{eqref}, чтобы не писать номера разделов и уравнений в
тексте явно.
\end{tblock}
\footnotetext{Фрагмент известной статьи, написанной генератором квазинаучных текстов
  (и принятой к публикации!),
  \href{https://ru.wikipedia.org/wiki/\%D0\%9A\%D0\%BE\%D1\%80\%D1\%87\%D0\%B5\%D0\%B2\%D0\%B0\%D1\%82\%D0\%B5\%D0\%BB\%D1\%8C_(\%D1\%81\%D1\%82\%D0\%B0\%D1\%82\%D1\%8C\%D1\%8F)}{\texttt{https://ru.wikipedia.org/wiki/Корчеватель\_(статья)}}.}
\vskip 2ex
\begin{center}
\ovlhref{structure-exercise.tex}{Щёлкните, чтобы открыть упражнение в \wllogo{}}
\end{center}
\begin{itemize}
\item Когда попробуете,
\ovlhref{structure-exercise-solution.tex}{щёлкните, чтобы посмотреть решение}.
\end{itemize}
\vspace{5pt}
\end{frame}

%%%%%%%%%%%%%%%%%%%%%%%%%%%%%%%%%%%%%%%%%%%%%%%%%%%%%%%%%%%%%%%%%%%%%%%%%%%%%%%
%%%%%%%%%%%%%%%%%%%%%%%%%%%%%%%%%%%%%%%%%%%%%%%%%%%%%%%%%%%%%%%%%%%%%%%%%%%%%%%
%%%%%%%%%%%%%%%%%%%%%%%%%%%%%%%%%%%%%%%%%%%%%%%%%%%%%%%%%%%%%%%%%%%%%%%%%%%%%%%
\section{Рисунки и таблицы}

%%%%%%%%%%%%%%%%%%%%%%%%%%%%%%%%%%%%%%%%%%%%%%%%%%%%%%%%%%%%%%%%%%%%%%%%%%%%%%%
%%%%%%%%%%%%%%%%%%%%%%%%%%%%%%%%%%%%%%%%%%%%%%%%%%%%%%%%%%%%%%%%%%%%%%%%%%%%%%%
%%%%%%%%%%%%%%%%%%%%%%%%%%%%%%%%%%%%%%%%%%%%%%%%%%%%%%%%%%%%%%%%%%%%%%%%%%%%%%%
\begin{frame}
\frametitle{План}
\vspace{-3ex}
\begin{multicols}{2}
\tableofcontents[currentsection]
\end{multicols}
\end{frame}

%%%%%%%%%%%%%%%%%%%%%%%%%%%%%%%%%%%%%%%%%%%%%%%%%%%%%%%%%%%%%%%%%%%%%%%%%%%%%%%
%%%%%%%%%%%%%%%%%%%%%%%%%%%%%%%%%%%%%%%%%%%%%%%%%%%%%%%%%%%%%%%%%%%%%%%%%%%%%%%
%%%%%%%%%%%%%%%%%%%%%%%%%%%%%%%%%%%%%%%%%%%%%%%%%%%%%%%%%%%%%%%%%%%%%%%%%%%%%%%
\subsection{Иллюстрации}

\begin{frame}[fragile]
\frametitle{\insertsubsection}
\begin{itemize}
\item Требуют пакет \bftt{graphicx}, в котором предоставляется команда
\cmdbs{includegraphics}.
\item Поддерживаемые графические форматы включают JPEG, PNG и PDF (обычно).
\end{itemize}
\begin{exampletwouptiny}
\includegraphics[
  width=0.7\textwidth]{gerbil}

\includegraphics[
  width=0.5\textwidth,
  angle=270]{gerbil}
\end{exampletwouptiny}

\tiny{Лицензия на изображение: \href{https://pixabay.com/en/animal-apple-attractive-beautiful-1239390/}{CC0}}
\end{frame}

%%%%%%%%%%%%%%%%%%%%%%%%%%%%%%%%%%%%%%%%%%%%%%%%%%%%%%%%%%%%%%%%%%%%%%%%%%%%%%%
%%%%%%%%%%%%%%%%%%%%%%%%%%%%%%%%%%%%%%%%%%%%%%%%%%%%%%%%%%%%%%%%%%%%%%%%%%%%%%%
%%%%%%%%%%%%%%%%%%%%%%%%%%%%%%%%%%%%%%%%%%%%%%%%%%%%%%%%%%%%%%%%%%%%%%%%%%%%%%%
\begin{frame}[fragile]
\frametitle{Отступление: Необязательные аргументы}
\vspace{-2ex}
\begin{itemize}
\item Для указания необязательных аргументов вместо скобок фигурных
\keystrokebftt{\{} \keystrokebftt{\}} используются скобки квадратные
\keystrokebftt{[} \keystrokebftt{]}.
\item \cmdbs{includegraphics} принимает необязательные аргументы, которые
позволяют преобразовать рисунок при включении его в документ. Например,
\bftt{width=0.3\cmdbs{textwidth}} меняет размеры рисунка так, чтобы его
ширина была равна 30\% ширины окружающего его текста (\cmdbs{textwidth}).
\item \cmdbs{documentclass} тоже принимает необязательные аргументы.
Например,
\mint{latex}|\documentclass[12pt,twocolumn]{article}|
увеличит шрифт текста (12pt) и инструктирует \LaTeX{} размещать его в две
колонки.
\item Где вам узнать про все эти возможности? В самом конце этой
презентации содержатся ссылки на ресурсы с полезной информацией.
\end{itemize}
\end{frame}

%%%%%%%%%%%%%%%%%%%%%%%%%%%%%%%%%%%%%%%%%%%%%%%%%%%%%%%%%%%%%%%%%%%%%%%%%%%%%%%
%%%%%%%%%%%%%%%%%%%%%%%%%%%%%%%%%%%%%%%%%%%%%%%%%%%%%%%%%%%%%%%%%%%%%%%%%%%%%%%
%%%%%%%%%%%%%%%%%%%%%%%%%%%%%%%%%%%%%%%%%%%%%%%%%%%%%%%%%%%%%%%%%%%%%%%%%%%%%%%
\subsection{Плавающие объекты}

\begin{frame}
\frametitle{\insertsubsection}
\vspace{-2ex}
\begin{itemize}
\item Позвольте \LaTeX{} решить, куда поставить рисунок (он может <<плавать>>).
\item Также можно снабдить рисунок номером и подписью, на которые потом можно
  ссылаться с помощью \cmdbs{ref}.
\end{itemize}
%\vspace{-6ex}
\begin{minipage}[t]{0.56\linewidth}
\inputcode[minted options={fontsize=\scriptsize}]{media-graphics.tex}
\end{minipage}~~%
\begin{minipage}[t]{0.40\linewidth}
\includegraphics[width=\textwidth,clip,viewport=0cm 242mm 4.3cm 297mm]{media-graphics.pdf}
\end{minipage}
\end{frame}

%%%%%%%%%%%%%%%%%%%%%%%%%%%%%%%%%%%%%%%%%%%%%%%%%%%%%%%%%%%%%%%%%%%%%%%%%%%%%%%
%%%%%%%%%%%%%%%%%%%%%%%%%%%%%%%%%%%%%%%%%%%%%%%%%%%%%%%%%%%%%%%%%%%%%%%%%%%%%%%
%%%%%%%%%%%%%%%%%%%%%%%%%%%%%%%%%%%%%%%%%%%%%%%%%%%%%%%%%%%%%%%%%%%%%%%%%%%%%%%
\subsection{Таблицы}

\begin{frame}[fragile]
\frametitle{\insertsubsection}
\vspace{-3ex}
\begin{itemize}\small
\item Чтобы привыкнуть к таблицам в \LaTeX{}, нужно время.
\item Можно воспользоваться окружением \bftt{tabular}.
\item Аргумент задаёт выравнивание колонок --- \textbf{l}eft, \textbf{r}ight, \textbf{r}ight.
\begin{exampletwouptiny}
\begin{tabular}{lrr}
Прод.  & \# & Цена \$ \\
Гаджет & 1  & 199.99  \\
Виджет & 2  & 399.99  \\
\end{tabular}
\end{exampletwouptiny}
\vspace{-1ex}
\item В нём также задаются вертикальные линии; команда \cmdbs{hline} проводит горизонтальную линию.
\begin{exampletwouptiny}
\begin{tabular}{|l|r|r|} \hline
Прод.  & \# & Цена \$ \\\hline
Виджет & 1  & 199.99  \\
Гаджет & 2  & 399.99  \\\hline
\end{tabular}
\end{exampletwouptiny}
\item Амперсанд \keystrokebftt{\&} разделяет колонки, а двойная обратная косая
  \keystrokebftt{\bs}\keystrokebftt{\bs} начинает новую строку (как в окружении
  \bftt{align*} из первой \rlap{части).}
\end{itemize}
\end{frame}

%%%%%%%%%%%%%%%%%%%%%%%%%%%%%%%%%%%%%%%%%%%%%%%%%%%%%%%%%%%%%%%%%%%%%%%%%%%%%%%
%%%%%%%%%%%%%%%%%%%%%%%%%%%%%%%%%%%%%%%%%%%%%%%%%%%%%%%%%%%%%%%%%%%%%%%%%%%%%%%
%%%%%%%%%%%%%%%%%%%%%%%%%%%%%%%%%%%%%%%%%%%%%%%%%%%%%%%%%%%%%%%%%%%%%%%%%%%%%%%
\addtocontents{toc}{\protect\pagebreak}
\section{Библиография}

%%%%%%%%%%%%%%%%%%%%%%%%%%%%%%%%%%%%%%%%%%%%%%%%%%%%%%%%%%%%%%%%%%%%%%%%%%%%%%%
%%%%%%%%%%%%%%%%%%%%%%%%%%%%%%%%%%%%%%%%%%%%%%%%%%%%%%%%%%%%%%%%%%%%%%%%%%%%%%%
%%%%%%%%%%%%%%%%%%%%%%%%%%%%%%%%%%%%%%%%%%%%%%%%%%%%%%%%%%%%%%%%%%%%%%%%%%%%%%%
\begin{frame}
\frametitle{План}
\vspace{-3ex}
\begin{multicols}{2}
\tableofcontents[currentsection]
\end{multicols}
\end{frame}

%%%%%%%%%%%%%%%%%%%%%%%%%%%%%%%%%%%%%%%%%%%%%%%%%%%%%%%%%%%%%%%%%%%%%%%%%%%%%%%
%%%%%%%%%%%%%%%%%%%%%%%%%%%%%%%%%%%%%%%%%%%%%%%%%%%%%%%%%%%%%%%%%%%%%%%%%%%%%%%
%%%%%%%%%%%%%%%%%%%%%%%%%%%%%%%%%%%%%%%%%%%%%%%%%%%%%%%%%%%%%%%%%%%%%%%%%%%%%%%
\subsection{Biblatex}

\begin{frame}[fragile]
\frametitle{\insertsubsection{} 1}
\vspace{-2ex}
\begin{itemize}
\item Сохраните ваши ссылки в \bftt{.bib} файле формата <<bibtex>>:
\inputbibtexcode[minted options={fontsize=\scriptsize}]{bib-example.bib}
\item Большинство менеджеров библиографических ссылок могут экспортировать в
  формат bibtex.
\end{itemize}
\end{frame}

%%%%%%%%%%%%%%%%%%%%%%%%%%%%%%%%%%%%%%%%%%%%%%%%%%%%%%%%%%%%%%%%%%%%%%%%%%%%%%%
%%%%%%%%%%%%%%%%%%%%%%%%%%%%%%%%%%%%%%%%%%%%%%%%%%%%%%%%%%%%%%%%%%%%%%%%%%%%%%%
%%%%%%%%%%%%%%%%%%%%%%%%%%%%%%%%%%%%%%%%%%%%%%%%%%%%%%%%%%%%%%%%%%%%%%%%%%%%%%%
\begin{frame}[fragile]
\frametitle{\insertsubsection{} 2}
\begin{itemize}
\item Каждый элемент в \bftt{.bib} файле включает \emph{ключ}, с помощью которого
на него можно ссылаться в документе. Например, \bftt{semenov2003paradox} это
ключ для этой статьи:
\begin{bibtexcode}
@article{semenov2003paradox,
  author = {Семёнов, Владимир},
}
\end{bibtexcode}
\item Хорошие ключи получаются из комбинаций имени автора, года и названия.
\item \LaTeX{} с помощью пакета \bftt{biblatex} автоматически отформатирует
цитаты в тексте и сгенерирует список литературы; он поддерживает большинство
стандартных стилей, но вы всегда можете разработать свой собственный.
\end{itemize}
\end{frame}

%%%%%%%%%%%%%%%%%%%%%%%%%%%%%%%%%%%%%%%%%%%%%%%%%%%%%%%%%%%%%%%%%%%%%%%%%%%%%%%
%%%%%%%%%%%%%%%%%%%%%%%%%%%%%%%%%%%%%%%%%%%%%%%%%%%%%%%%%%%%%%%%%%%%%%%%%%%%%%%
%%%%%%%%%%%%%%%%%%%%%%%%%%%%%%%%%%%%%%%%%%%%%%%%%%%%%%%%%%%%%%%%%%%%%%%%%%%%%%%
\begin{frame}[fragile]
\frametitle{\insertsubsection{} 3}
\vspace{-3ex}
\small
\begin{itemize}
\item Используйте пакет \bftt{biblatex}\footnote{\scriptsize Многие шаблоны статей используют
\bftt{natbib} вместе с \bftt{bibtex}. Но эта связка не поддерживает русский язык.}
и команды \cmdbs{cite} and \cmdbs{parencite}.
\item Включите файл со ссылками через \cmdbs{addbibresource} и выведите список
литературы через \cmdbs{printbibliography}.
\end{itemize}
\vspace{-2ex}
\begin{minipage}{0.55\linewidth}
\inputcode[minted options={fontsize=\scriptsize}]{bib-example.tex}
\end{minipage}~~%
\begin{minipage}{0.45\linewidth}
\includegraphics[width=\textwidth,clip,viewport=0cm 232mm 5.2cm 297mm]{bib-example.pdf}
\end{minipage}
\end{frame}

%%%%%%%%%%%%%%%%%%%%%%%%%%%%%%%%%%%%%%%%%%%%%%%%%%%%%%%%%%%%%%%%%%%%%%%%%%%%%%%
%%%%%%%%%%%%%%%%%%%%%%%%%%%%%%%%%%%%%%%%%%%%%%%%%%%%%%%%%%%%%%%%%%%%%%%%%%%%%%%
%%%%%%%%%%%%%%%%%%%%%%%%%%%%%%%%%%%%%%%%%%%%%%%%%%%%%%%%%%%%%%%%%%%%%%%%%%%%%%%
\subsection{Упражнение}

\begin{frame}[fragile]
\frametitle{Упражнение: соберите всё вместе}
Вставьте рисунок и список литературы в статью из предыдущего упражнения.
\begin{enumerate}
\item Загрузите эти два файла на ваш компьютер.
\vspace{1ex}
\begin{center}
\filhref{gerbil.jpg?dl=1}{Щёлкните, чтобы загрузить пример рисунка}

\filhref{bib-exercise.bib?dl=1}{Щёлкните, чтобы загрузить приммер bib файла}
\end{center}
\vspace{1ex}
\item Загрузите их на Overleaf (используйте меню проекта).
\item Когда попробуете,
\ovlhref{structure-exercise-full-solution.tex}{щёлкните, чтобы посмотреть решение}.
\end{enumerate}
\end{frame}

%%%%%%%%%%%%%%%%%%%%%%%%%%%%%%%%%%%%%%%%%%%%%%%%%%%%%%%%%%%%%%%%%%%%%%%%%%%%%%%
%%%%%%%%%%%%%%%%%%%%%%%%%%%%%%%%%%%%%%%%%%%%%%%%%%%%%%%%%%%%%%%%%%%%%%%%%%%%%%%
%%%%%%%%%%%%%%%%%%%%%%%%%%%%%%%%%%%%%%%%%%%%%%%%%%%%%%%%%%%%%%%%%%%%%%%%%%%%%%%
\section{Что дальше?}

%%%%%%%%%%%%%%%%%%%%%%%%%%%%%%%%%%%%%%%%%%%%%%%%%%%%%%%%%%%%%%%%%%%%%%%%%%%%%%%
%%%%%%%%%%%%%%%%%%%%%%%%%%%%%%%%%%%%%%%%%%%%%%%%%%%%%%%%%%%%%%%%%%%%%%%%%%%%%%%
%%%%%%%%%%%%%%%%%%%%%%%%%%%%%%%%%%%%%%%%%%%%%%%%%%%%%%%%%%%%%%%%%%%%%%%%%%%%%%%
\begin{frame}
\frametitle{План}
\vspace{-3ex}
\begin{multicols}{2}
\tableofcontents[currentsection]
\end{multicols}
\end{frame}

%%%%%%%%%%%%%%%%%%%%%%%%%%%%%%%%%%%%%%%%%%%%%%%%%%%%%%%%%%%%%%%%%%%%%%%%%%%%%%%
%%%%%%%%%%%%%%%%%%%%%%%%%%%%%%%%%%%%%%%%%%%%%%%%%%%%%%%%%%%%%%%%%%%%%%%%%%%%%%%
%%%%%%%%%%%%%%%%%%%%%%%%%%%%%%%%%%%%%%%%%%%%%%%%%%%%%%%%%%%%%%%%%%%%%%%%%%%%%%%
\subsection{Ещё больше изящных возможностей}

\begin{frame}[fragile]
\frametitle{\insertsubsection}
\vspace{-2ex}
\begin{itemize}
\item Вставьте команду \cmdbs{tableofcontents}, и она сгенерирует оглавление,
собрав заголовки, введённые командами \cmdbs{section} и \cmdbs{subsection}.
\item Измените \cmdbs{documentclass} на
\mint{latex}!\documentclass{scrartcl}!
или
\mint{latex}!\documentclass[12pt]{IEEEtran}!
\item Определите свои собственные команды, чтобы избежать повторения набора
  сложных конструкций:
\begin{exampletwouptiny}
\newcommand{\rperf}{%
  \rho_{\text{perf}}}
\[
\rperf = \mathbf{c}'\mathbf{X}
       + \varepsilon
\]
\end{exampletwouptiny}
\end{itemize}
\end{frame}

%%%%%%%%%%%%%%%%%%%%%%%%%%%%%%%%%%%%%%%%%%%%%%%%%%%%%%%%%%%%%%%%%%%%%%%%%%%%%%%
%%%%%%%%%%%%%%%%%%%%%%%%%%%%%%%%%%%%%%%%%%%%%%%%%%%%%%%%%%%%%%%%%%%%%%%%%%%%%%%
%%%%%%%%%%%%%%%%%%%%%%%%%%%%%%%%%%%%%%%%%%%%%%%%%%%%%%%%%%%%%%%%%%%%%%%%%%%%%%%
\subsection{Ещё больше изящных пакетов}

\begin{frame}
\frametitle{\insertsubsection}
\begin{itemize}
  \item \bftt{beamer}: для презентаций (таких как эта!)
\item \bftt{todonotes}: комментарии и отметки <<что ещё сделать>>
\item \bftt{tikz}: потрясающая графическая библиотека
\item \bftt{pgfplots}: создание графиков в \LaTeX
\item \bftt{listings}: вывод листингов программного кода для \LaTeX
\item \bftt{spreadtab}: создание таблиц в \LaTeX
\item \bftt{gchords}, \bftt{guitar}: гитарные аккорды и табулатуры
\item \bftt{cwpuzzle}: кроссворды
\end{itemize}
Примеры применения этих и других пакетов можно найти на
\url{https://www.overleaf.com/latex/examples} и \url{http://texample.net}.
\end{frame}

%%%%%%%%%%%%%%%%%%%%%%%%%%%%%%%%%%%%%%%%%%%%%%%%%%%%%%%%%%%%%%%%%%%%%%%%%%%%%%%
%%%%%%%%%%%%%%%%%%%%%%%%%%%%%%%%%%%%%%%%%%%%%%%%%%%%%%%%%%%%%%%%%%%%%%%%%%%%%%%
%%%%%%%%%%%%%%%%%%%%%%%%%%%%%%%%%%%%%%%%%%%%%%%%%%%%%%%%%%%%%%%%%%%%%%%%%%%%%%%
\subsection{Установка \LaTeX{}}

\begin{frame}
\frametitle{\insertsubsection}
\begin{itemize}
\item Для запуска \LaTeX{} на вашем собственном компьютере вам понадобится установить
\emph{дистрибутив} \LaTeX{}. Типичный дистрибутив включает в себя программу \bftt{latex}
(скорее понадобится \bftt{pdflatex} или \bftt{lualatex}) и несколько тысяч пакетов.
\begin{itemize}
\item Для Windows: \href{http://miktex.org/}{Mik\TeX} или \href{http://tug.org/texlive/}{\TeX Live}
\item Для Linux: \href{http://tug.org/texlive/}{\TeX Live}
\item Для MacOS X: \href{http://tug.org/mactex/}{Mac\TeX}
\end{itemize}
\item Также вам понадобится текстовый редактор с поддержкой \LaTeX{}. Множество
вариантов можно посмотреть по ссылке
\href{http://en.wikipedia.org/wiki/Comparison_of_TeX_editors}%
{\mbox{\texttt{http://en.wikipedia.org/wiki/Comparison\_of\_TeX\_editors}}}.
\item Вам также понадобится информация о том, как именно запускать \bftt{latex},
как он и сопутствующие программы работают. На следующем слайде приведён список
полезных информационных ресурсов.
\end{itemize}
\end{frame}

%%%%%%%%%%%%%%%%%%%%%%%%%%%%%%%%%%%%%%%%%%%%%%%%%%%%%%%%%%%%%%%%%%%%%%%%%%%%%%%
%%%%%%%%%%%%%%%%%%%%%%%%%%%%%%%%%%%%%%%%%%%%%%%%%%%%%%%%%%%%%%%%%%%%%%%%%%%%%%%
%%%%%%%%%%%%%%%%%%%%%%%%%%%%%%%%%%%%%%%%%%%%%%%%%%%%%%%%%%%%%%%%%%%%%%%%%%%%%%%
\subsection{Онлайновые ресурсы}

\begin{frame}
\frametitle{\insertsubsection}
\begin{itemize}
\item \genhref{https://www.overleaf.com/learn}{Вики проекта Overleaf} ---
обучающие материалы проекта Overleaf
\item \genhref{http://en.wikibooks.org/wiki/LaTeX}{Викибук про \LaTeX{}} ---
отличные вводные и справочные материалы.
\item \genhref{http://tex.stackexchange.com/}{\TeX{} Stack Exchange} --- задайте
вопрос и получите исчерпывающий ответ, причём очень быстро.
\item \genhref{http://www.latex-community.org/}{Сообщество \LaTeX{}} --- большой
онлайновый форум.
\item \genhref{http://ctan.org/}{Всеобъемлющая сеть архивов \TeX{} (CTAN)} ---
более четырёх тысяч пакетов плюс документация.
\item Google обычно отсылает на один из перечисленных ресурсов.
\end{itemize}
\end{frame}

%%%%%%%%%%%%%%%%%%%%%%%%%%%%%%%%%%%%%%%%%%%%%%%%%%%%%%%%%%%%%%%%%%%%%%%%%%%%%%%
%%%%%%%%%%%%%%%%%%%%%%%%%%%%%%%%%%%%%%%%%%%%%%%%%%%%%%%%%%%%%%%%%%%%%%%%%%%%%%%
%%%%%%%%%%%%%%%%%%%%%%%%%%%%%%%%%%%%%%%%%%%%%%%%%%%%%%%%%%%%%%%%%%%%%%%%%%%%%%%
\begin{frame}
\begin{center}
Спасибо, и приятного \TeX{}анья!
\end{center}
\end{frame}

\end{document}

% -- latex understands words, sentences and paragraphs

Words are separated by one or more spaces.  Paragraphs are separated by
one or more blank lines.  The output is not affected by adding extra
spaces or extra blank lines to the input file.

Double quotes are typed like this: ``quoted text''.
Single quotes are typed like this: `single-quoted text'.

Emphasized text is typed like this: \emph{this is emphasized}.
Bold       text is typed like this: \textbf{this is bold}.

-- Adding structure to your document

\section{Hello}

\subsection{World}

\subsection{Foo}

\subsubsection*{Stuff} % star form

\subsubsection*{Results}

-- Labels and cross-references

\label{sec:intro}
\label{sec:method}
\ref{sec:method}

--> maybe introduce the prettyref package here.

-- Mathematics

Inline mathematics: $x + y < 7$.

'Displayed' mathematics:
\begin{equation}
\end{equation}

\begin{equation*}
\end{equation*}

\begin{align}
\end{align}

-- Figures

- Need the graphicx package.

- here we can start introducing options

\includegraphics[width=\textwidth]{}

- where do you find out about these options? --> link to the Wikibook

-- Floating Figures

\begin{figure}
\includegraphics{...}
\caption{\label{}Here is a caption.}
\end{figure}

-- Tables

- not the nicest part of LaTeX

\usepackage{tabularx}

\begin{tabular}{llr}
Item & Quantity & Price (\$) & Amount
Widget & 1 &
\end{tabular}

Bonus points: check out the fp package and the spreadtab package.

-- Document Classes

a .cls file

article

some journal templates come with one

-- Bibliographies



-- For Typesetting Geeks

- dashes: -, --, ---

- ellipsis.

- controlling spaces: ~, \ , \,, \@

- spacing after periods (et al., etc.)

- Nested quotation marks: ``\,`
\vskip 2ex
\item Use the \emph{star form} to display an equation without a number.
\begin{exampletwouptiny}
\begin{equation*}
F(x) = \int_{a}^{x}{f(t) dt}
\end{equation*}
\end{exampletwouptiny}

\begin{itemize}
\item \bftt{equation} and \bftt{equation*} are called \emph{environments}.
\begin{itemize}
  \item The \cmdbs{begin} and \cmdbs{end} commands define the environment.
  \item The \cmd{\$} also starts and ends an environment.
  \item Some commands are defined only within certain environments.
  \item Some commands behave differently in different environments.
\end{itemize}
\end{itemize}
\end{block}
\begin{center}
\fbox{\href{http://ctan.org/}{The Comprehensive \TeX Archive Network (CTAN)}}
\end{center}

%%%%%%%%%%%%%%%%%%%%%%%%%%%%%%%%%%%%%%%%%%%%%%%%%%%%%%%%%%%%%%%%%%%%%%%%%%%%%%%
%%%%%%%%%%%%%%%%%%%%%%%%%%%%%%%%%%%%%%%%%%%%%%%%%%%%%%%%%%%%%%%%%%%%%%%%%%%%%%%
%%%%%%%%%%%%%%%%%%%%%%%%%%%%%%%%%%%%%%%%%%%%%%%%%%%%%%%%%%%%%%%%%%%%%%%%%%%%%%%
\subsection{Typography tweaks}
\begin{frame}{\insertsubsection}
\begin{tabular}{lll}
& character name & used mainly for \ldots \\\hline
\bftt{\bs} & backslash                 & commands, tables \\
\bftt{\{}  & open brace                & commands \\
\bftt{\}}  & close brace               & commands \\
\bftt{\%}  & percent sign              & comments \\
\bftt{\#}  & hash (pound / sharp) sign & custom commands \\
\bftt{\$}  & dollar sign               & equations \\
\bftt{\_}  & underscore                & equations (subscripts) \\
\bftt{\^}  & caret                     & equations (superscripts) \\
\bftt{\&}  & ampersand                 & tables \\
\bftt{\~}  & tilde                     & spacing \\
\end{tabular}
\end{frame}

%\item We've used several environments:
%\vskip 1ex
%{\scriptsize
%\begin{tabular}{ll}
%\cmdbs{begin}\bftt{\{document\}}\ldots\cmdbs{end}\bftt{\{document\}} &
%  document environment \\
%\cmdbs{begin}\bftt{\{itemize\}}\ldots\cmdbs{end}\bftt{\{itemize\}} &
%  itemized list environment \\
%\bftt{\$\ldots\$}     & \emph{in-text} math environment \\
%\bftt{\$\$\ldots\$\$} & \emph{displayed} math environment \\
%\cmdbs{begin}\bftt{\{equation\}}\ldots\cmdbs{end}\bftt{\{equation\}} &
%  displayed math environment w/ number
%\end{tabular}
%}
