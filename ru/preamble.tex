%
% Common preamble for all three parts.
%

\usetheme[block=fill]{metropolis}
\setbeamercolor{frametitle}{use=normal text, parent=normal text}

\usepackage{arevmath}
\SetSymbolFont{largesymbols}{normal}{OMX}{iwona}{m}{n}
\usepackage{fontspec}
\setmainfont{PT Sans}
\setsansfont{PT Sans}
\setmonofont{PT Mono}[Scale=0.87]
\usepackage[english,russian]{babel}
\usepackage{amsmath}
\usepackage{color}
\usepackage{minted}
\usepackage{hyperref}
\usepackage{multicol}
\usepackage{tabularx}
\usepackage{tikz}
\usepackage{tcolorbox}

\hypersetup{unicode=true}

\tcbuselibrary{skins}
\tcbuselibrary{listings}
\tcbuselibrary{minted}
\tcbset{colframe=mDarkTeal, colback=white!90!mDarkTeal,% Taken from the Metropolis theme
        left=0.8em,right=0.8em}
\newtcolorbox{tblock}[1]{boxsep=1mm,sidebyside=false,bicolor=false,colback=white,title={#1}}
\newtcolorbox{printout}{boxsep=0mm,sidebyside=false,bicolor=false,colback=white}
\def\ovllink#1{\tcbox[on line,boxsep=0mm,left=2pt,right=2pt,top=2pt,bottom=2pt,
                      colback=white]{#1}}
\newtcblisting{code}{boxsep=0mm,listing only,minted language=latex}
\newtcblisting{bibtexcode}{boxsep=0mm,listing only,minted language=bibtex}
\newtcblisting{exampletwoup}{fontupper=\small,fontlower=\small,
                             boxsep=0mm,listing side text,minted language=latex,
                             bicolor,colbacklower=white,
                             righthand ratio=0.42}
\newtcblisting{exampletwouptiny}{fontupper=\footnotesize,fontlower=\footnotesize,
                                 boxsep=0mm,listing side text,minted language=latex,
                                 minted options={fontsize=\footnotesize},
                                 bicolor,colbacklower=white,
                                 righthand ratio=0.42}
\newtcbinputlisting{\inputcode}[2][]{%
listing file={#2},boxsep=0mm,listing only,minted language=latex,#1}
\newtcbinputlisting{\inputbibtexcode}[2][]{%
listing file={#2},boxsep=0mm,listing only,minted language=bibtex,#1}

% only inline todonotes work
\usepackage{xkeyval}
\usepackage[textsize=small]{todonotes}
\presetkeys{todonotes}{inline}{}

\usetikzlibrary{shapes,arrows,positioning,shadows}

% no nav buttons
\usenavigationsymbolstemplate{}

\newcommand{\bftt}[1]{\textbf{\texttt{#1}}}
%\newcommand{\comment}[1]{{\color[HTML]{008080}\textit{\textbf{\texttt{#1}}}}}
\newcommand{\cmd}[1]{{\color[HTML]{008000}\bftt{#1}}}
\newcommand{\bs}{\char`\\}
\newcommand{\cmdbs}[1]{\cmd{\bs#1}}
\newcommand{\lcb}{\char '173}
\newcommand{\rcb}{\char '175}
\newcommand{\cmdbegin}[1]{\cmdbs{begin\lcb}\bftt{#1}\cmd{\rcb}}
\newcommand{\cmdend}[1]{\cmdbs{end\lcb}\bftt{#1}\cmd{\rcb}}

\newcommand{\wllogo}{\textbf{Overleaf}}

% this is where the example source files are loaded from
% do not include a trailing slash
\newcommand{\fileuri}{https://raw.github.com/sgolovan/latex-course/master/ru}

\newcommand{\wlserver}{https://www.overleaf.com}
\newcommand{\wlnewdoc}[1]{\wlserver/docs?snip\_uri=\fileuri/#1\&splash=none}

\def\tikzname{Ti\emph{k}Z}

% from http://tex.stackexchange.com/questions/5226/keyboard-font-for-latex
\newcommand*\keystroke[1]{%
  \tikz[baseline=(key.base)]
    \node[%
      draw,
      fill=white,
      drop shadow={shadow xshift=0.25ex,shadow yshift=-0.25ex,fill=black,opacity=0.75},
      rectangle,
      rounded corners=2pt,
      inner sep=1pt,
      line width=0.5pt,
      font=\scriptsize\ttfamily
    ](key) {#1\strut}
  ;
}
\newcommand{\keystrokebftt}[1]{\keystroke{\bftt{#1}}}

% stolen from minted.dtx
\newenvironment{exampletwouptinynoframe}
  {\VerbatimEnvironment
   \begin{VerbatimOut}{example.out}}
  {\end{VerbatimOut}
   \setlength{\parindent}{0pt}
   \begin{tabular}{l|l}
   \begin{minipage}{0.55\linewidth}
     \inputminted[fontsize=\scriptsize,resetmargins]{latex}{example.out}
   \end{minipage} &
   \begin{minipage}{0.35\linewidth}
     \setlength{\parskip}{6pt plus 1pt minus 1pt}%
     \raggedright\scriptsize\input{example.out}
   \end{minipage}
   \end{tabular}}

\title{Интерактивное введение в \LaTeX}
\author{Джон Д. Лис-Миллер\\Перевод на русский язык: Сергей Головань}
%\titlegraphic{%
%\includegraphics[height=36pt]{overleaf}\\[1em]
%\includegraphics[height=24pt]{UoB-logo}
%}
