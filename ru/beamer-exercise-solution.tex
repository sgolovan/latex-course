\documentclass{beamer}

%
% Choose how your presentation looks.
%
% For more themes, color themes and font themes, see:
% http://deic.uab.es/~iblanes/beamer_gallery/index_by_theme.html
%
\mode<presentation>
{
  \usetheme{Boadilla} % or try Darmstadt, Madrid, Dresden, ...
  \usecolortheme{seagull} % or try albatross, beaver, crane, ...
  \usefonttheme{default} % or try serif, structurebold, ...
  \setbeamertemplate{navigation symbols}{}
  \setbeamertemplate{caption}[numbered]
} 

\usepackage{cmbright} % Математика "без засечек"
\usepackage{fontspec}
\setsansfont{CMU Sans Serif}
\usepackage[russian]{babel}

\usepackage{pgfplots} % Диаграмма

\hypersetup{unicode=true} % Кириллица в закладках PDF

\title{Освящение Геттисбергского кладбища}
\author{Авраам Линкольн}
\institute[США]{Соединённые Штаты Америки}
\date{19 ноября 1863 г.}

\begin{document}

\begin{frame}
  \titlepage
\end{frame}

\begin{frame}
\frametitle{План}
  \tableofcontents
\end{frame}

\section{Повестка дня}

\begin{frame}
\frametitle{Повестка дня}
\begin{itemize}
  \item Встретились на поле битвы (великом)
  \item Освятить часть поля --- уместно!
  \item Неоконченное дело (великие задачи)
\end{itemize}
\end{frame}

\begin{frame}
\frametitle{Невозможно включить в повестку дня!}
\begin{itemize}[<+->]
  \item Освятить
  \item Сделать священной
  \item Одухотворить (в узком смысле)
  \item Прибавить или убавить
  \item Заметить или запомнить, что было сказано
\end{itemize}
\end{frame}

\section{Обзор}

\begin{frame}
\frametitle{Ключевые задачи \& составляющие успеха}
\begin{itemize}
\item Что делает нацию уникальной
  \begin{itemize}
    \item Рождённая в свободе
    \item Все люди равны
  \end{itemize}
\end{itemize}

\begin{block}{Общее видение}
\begin{itemize}
\item Возрождение свободы.
\item Власть народа, волей народа и для народа.
\end{itemize}
\end{block}
\end{frame}

\begin{frame}
\frametitle{Организационный обзор}
\begin{figure}
\begin{tikzpicture}
\begin{axis}[
  width=7cm,
  height=5cm,
  symbolic x coords={$-87$ лет, сейчас},
  xtick=data,
  nodes near coords,
  ybar=6pt,
  enlargelimits=0.2,
  every axis plot/.append style=thick,
  axis line style = thick,
  legend style={draw=normal text.fg,fill=normal text.bg}
  ]
\addplot[block title.fg,fill=block title.bg] coordinates {
  ($-87$ лет,   1)
  (сейчас,  0)
};
\legend{Новые нации};
\end{axis}
\end{tikzpicture}
\end{figure}

\begin{block}{Четыре двадцатилетия и семь}
\[
-(4 \times 20 + 7) = -87
\]
\end{block}
\end{frame}

\section{Резюме}

\begin{frame}
\frametitle{Резюме}
\begin{columns}
\begin{column}{0.4\textwidth}
\begin{itemize}
\item Новая нация
\item Гражданская война
\item Освятить поле
\end{itemize}
\end{column}
\begin{column}{0.6\textwidth}
\begin{itemize}
\item Посвятим себя неоконченному делу
\item Возрождение свободы
\item Нация не исчезнет с лица земли
\end{itemize}
\end{column}
\end{columns}
\end{frame}

\end{document}
