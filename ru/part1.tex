\documentclass{beamer}

%
% Common preamble for all three parts.
%

\usetheme[block=fill]{metropolis}
\setbeamercolor{frametitle}{use=normal text, parent=normal text}

\usepackage{arevmath}
\SetSymbolFont{largesymbols}{normal}{OMX}{iwona}{m}{n}
\usepackage{fontspec}
\setmainfont{PT Sans}
\setsansfont{PT Sans}
\setmonofont{PT Mono}[Scale=0.87]
\usepackage[english,russian]{babel}
\usepackage{amsmath}
\usepackage{color}
\usepackage{minted}
\usepackage{hyperref}
\usepackage{multicol}
\usepackage{tabularx}
\usepackage{tikz}
\usepackage{tcolorbox}
\usepackage{xparse}

% For slide 28 (Tikz examples)
\usetikzlibrary{mindmap,trees}
\usetikzlibrary{backgrounds,shapes,arrows,positioning,calc,snakes,fit}
\usepgflibrary{decorations.markings}

\hypersetup{unicode=true}

\tcbuselibrary{skins}
\tcbuselibrary{listings}
\tcbuselibrary{minted}
\tcbset{colframe=mDarkTeal, colback=white!90!mDarkTeal,% Taken from the Metropolis theme
        left=0.8em,right=0.8em}
\newtcolorbox{tblock}[1]{boxsep=1mm,sidebyside=false,bicolor=false,colback=white,title={#1}}
\newtcolorbox{printout}{boxsep=0mm,sidebyside=false,bicolor=false,colback=white}
\def\linkbox#1{\tcbox[on line,boxsep=0mm,left=2pt,right=2pt,top=2pt,bottom=2pt,
                      colback=mDarkTeal,coltext=white]{#1}}
\def\ovllink#1{\tcbox[on line,boxsep=0mm,left=2pt,right=2pt,top=2pt,bottom=2pt,
                      colback=green!30!black,colframe=green!30!black,coltext=white]{#1}}
\newtcblisting{code}{boxsep=0mm,listing only,minted language=latex}
\newtcblisting{bibtexcode}{boxsep=0mm,listing only,minted language=bibtex}
\newtcblisting{exampletwoup}{fontupper=\small,fontlower=\small,
                             boxsep=0mm,listing side text,minted language=latex,
                             bicolor,colbacklower=white,
                             righthand ratio=0.42}
\newtcblisting{exampletwouptiny}{fontupper=\footnotesize,fontlower=\footnotesize,
                                 boxsep=0mm,listing side text,minted language=latex,
                                 minted options={fontsize=\footnotesize},
                                 bicolor,colbacklower=white,
                                 righthand ratio=0.42}
\newtcblisting{exampletwouppaused}{fontupper=\footnotesize,fontlower=\footnotesize,
                             boxsep=0mm,listing side text,minted language=latex,
                             minted options={fontsize=\footnotesize},
                             bicolor,colbacklower=white,
                             righthand ratio=0.42,after lower=\onslide<1->}
\newtcbinputlisting{\inputcode}[2][]{%
listing file={#2},boxsep=0mm,listing only,minted language=latex,#1}
\newtcbinputlisting{\inputbibtexcode}[2][]{%
listing file={#2},boxsep=0mm,listing only,minted language=bibtex,#1}

\DeclareDocumentCommand{\linkbox}{O{mDarkTeal}m}{%
\tcbox[on line,
       boxsep=0mm,
       left=2pt,
       right=2pt,
       top=2pt,
       bottom=2pt,
       colframe=#1,
       colback=#1,
       coltext=white]{#2}}

\newcommand{\genhref}[2]{\href{#1}{\linkbox{#2}}}
\newcommand{\filhref}[2]{\href{\fileuri/#1}{\linkbox{#2}}}
\newcommand{\ovlhref}[2]{\href{\wlnewdoc{#1}}{\linkbox[green!30!black]{#2}}}

% only inline todonotes work
\usepackage{xkeyval}
\usepackage[textsize=small]{todonotes}
\presetkeys{todonotes}{inline}{}

\usetikzlibrary{shapes,arrows,positioning,shadows}

% no nav buttons
\usenavigationsymbolstemplate{}

\newcommand{\bftt}[1]{\textbf{\texttt{#1}}}
%\newcommand{\comment}[1]{{\color[HTML]{008080}\textit{\textbf{\texttt{#1}}}}}
\newcommand{\cmd}[1]{{\color[HTML]{008000}\bftt{#1}}}
\newcommand{\bs}{\char`\\}
\newcommand{\cmdbs}[1]{\cmd{\bs#1}}
\newcommand{\lcb}{\char '173}
\newcommand{\rcb}{\char '175}
\newcommand{\cmdbegin}[1]{\cmdbs{begin\lcb}\bftt{#1}\cmd{\rcb}}
\newcommand{\cmdend}[1]{\cmdbs{end\lcb}\bftt{#1}\cmd{\rcb}}

\newcommand{\wllogo}{\textbf{Overleaf}}

% this is where the example source files are loaded from
% do not include a trailing slash
\newcommand{\fileuri}{https://raw.github.com/sgolovan/latex-course/master/ru}

\newcommand{\wlserver}{https://www.overleaf.com}
\newcommand{\wlnewdoc}[1]{\wlserver/docs?snip\_uri=\fileuri/#1\&splash=none}

\def\tikzname{Ti\emph{k}Z}

% from http://tex.stackexchange.com/questions/5226/keyboard-font-for-latex
\newcommand*\keystroke[1]{%
  \tikz[baseline=(key.base)]
    \node[%
      draw,
      fill=white,
      drop shadow={shadow xshift=0.25ex,shadow yshift=-0.25ex,fill=black,opacity=0.75},
      rectangle,
      rounded corners=2pt,
      inner sep=1pt,
      line width=0.5pt,
      font=\scriptsize\ttfamily
    ](key) {#1\strut}
  ;
}
\newcommand{\keystrokebftt}[1]{\keystroke{\bftt{#1}}}

% stolen from minted.dtx
\newenvironment{exampletwouptinynoframe}
  {\VerbatimEnvironment
   \begin{VerbatimOut}{example.out}}
  {\end{VerbatimOut}
   \setlength{\parindent}{0pt}
   \begin{tabular}{l|l}
   \begin{minipage}{0.55\linewidth}
     \inputminted[fontsize=\scriptsize,resetmargins]{latex}{example.out}
   \end{minipage} &
   \begin{minipage}{0.35\linewidth}
     \setlength{\parskip}{6pt plus 1pt minus 1pt}%
     \raggedright\scriptsize\input{example.out}
   \end{minipage}
   \end{tabular}}

\title{Интерактивное введение в \LaTeX}
\author{Джон Д. Лис-Миллер\\Перевод на русский язык: Сергей Головань}
%\titlegraphic{%
%\includegraphics[height=36pt]{overleaf}\\[1em]
%\includegraphics[height=24pt]{UoB-logo}
%}


\subtitle{Часть 1: Основы}

\begin{document}

%%%%%%%%%%%%%%%%%%%%%%%%%%%%%%%%%%%%%%%%%%%%%%%%%%%%%%%%%%%%%%%%%%%%%%%%%%%%%%%
%%%%%%%%%%%%%%%%%%%%%%%%%%%%%%%%%%%%%%%%%%%%%%%%%%%%%%%%%%%%%%%%%%%%%%%%%%%%%%%
%%%%%%%%%%%%%%%%%%%%%%%%%%%%%%%%%%%%%%%%%%%%%%%%%%%%%%%%%%%%%%%%%%%%%%%%%%%%%%%
\begin{frame}
\titlepage
\end{frame}

%%%%%%%%%%%%%%%%%%%%%%%%%%%%%%%%%%%%%%%%%%%%%%%%%%%%%%%%%%%%%%%%%%%%%%%%%%%%%%%
%%%%%%%%%%%%%%%%%%%%%%%%%%%%%%%%%%%%%%%%%%%%%%%%%%%%%%%%%%%%%%%%%%%%%%%%%%%%%%%
%%%%%%%%%%%%%%%%%%%%%%%%%%%%%%%%%%%%%%%%%%%%%%%%%%%%%%%%%%%%%%%%%%%%%%%%%%%%%%%
\begin{frame}{Почему \LaTeX{}?}
\begin{itemize}
\item Он позволяет набирать красивые документы
\begin{itemize}
\item Особенно с математическими формулами
\end{itemize}
%
\item Он был создан учёными для учёных
\begin{itemize}
\item Большое и активное сообщество пользователей и разработчиков
\end{itemize}
%
\item Он мощный, вы можете расширять его
\begin{itemize}
\item Пакеты для статей, презентаций, таблиц, \ldots
\end{itemize}
\end{itemize}
\end{frame}

%%%%%%%%%%%%%%%%%%%%%%%%%%%%%%%%%%%%%%%%%%%%%%%%%%%%%%%%%%%%%%%%%%%%%%%%%%%%%%%
%%%%%%%%%%%%%%%%%%%%%%%%%%%%%%%%%%%%%%%%%%%%%%%%%%%%%%%%%%%%%%%%%%%%%%%%%%%%%%%
%%%%%%%%%%%%%%%%%%%%%%%%%%%%%%%%%%%%%%%%%%%%%%%%%%%%%%%%%%%%%%%%%%%%%%%%%%%%%%%
\begin{frame}[fragile]{Как он работает?}
\begin{itemize}
\item Вы набираете документ в формате \emph{обычного текста} с \cmd{командами},
  которые описывают его структуру и смысл.
\item Программа \texttt{latex} обрабатывает ваш текст и команды и преобразовывает
  его в красиво отформатированный документ.
\end{itemize}
\begin{center}
\begin{code}
Шла Саша по шоссе и сосала \emph{сушку}.
\end{code}
\tikz\node[single arrow,fill=gray,font=\ttfamily\bfseries,%
  rotate=270,xshift=-1em]{latex};
\begin{printout}
Шла Саша по шоссе и сосала \emph{сушку}.
\end{printout}
\end{center}
\end{frame}

%%%%%%%%%%%%%%%%%%%%%%%%%%%%%%%%%%%%%%%%%%%%%%%%%%%%%%%%%%%%%%%%%%%%%%%%%%%%%%%
%%%%%%%%%%%%%%%%%%%%%%%%%%%%%%%%%%%%%%%%%%%%%%%%%%%%%%%%%%%%%%%%%%%%%%%%%%%%%%%
%%%%%%%%%%%%%%%%%%%%%%%%%%%%%%%%%%%%%%%%%%%%%%%%%%%%%%%%%%%%%%%%%%%%%%%%%%%%%%%
\begin{frame}[fragile]{Ещё примеры команд и их результат\ldots}
\vspace{-0.5mm}
\begin{exampletwoup}
\begin{itemize}
\item Чай
\item Молоко
\item Печенье
\end{itemize}
\end{exampletwoup}
\vspace{-0.5mm}
\begin{exampletwoup}
\begin{figure}
\includegraphics{chick}
\end{figure}
\end{exampletwoup}
\vspace{-0.5mm}
\begin{exampletwoup}
\begin{equation}
\alpha + \beta + 1
\end{equation}
\end{exampletwoup}
\vspace{-3mm}
\tiny{Изображение взято с \url{http://www.andy-roberts.net/writing/latex/importing_images}}
\end{frame}

%%%%%%%%%%%%%%%%%%%%%%%%%%%%%%%%%%%%%%%%%%%%%%%%%%%%%%%%%%%%%%%%%%%%%%%%%%%%%%%
%%%%%%%%%%%%%%%%%%%%%%%%%%%%%%%%%%%%%%%%%%%%%%%%%%%%%%%%%%%%%%%%%%%%%%%%%%%%%%%
%%%%%%%%%%%%%%%%%%%%%%%%%%%%%%%%%%%%%%%%%%%%%%%%%%%%%%%%%%%%%%%%%%%%%%%%%%%%%%%
\begin{frame}[fragile]{Смена подхода}

\begin{itemize}
\item Используйте команды, чтобы описывать <<что это>>, а не <<как это выглядит>>.
\item Сосредоточьтесь на содержании.
\item Позвольте \LaTeX{} делать его работу.
\end{itemize}
\end{frame}

%%%%%%%%%%%%%%%%%%%%%%%%%%%%%%%%%%%%%%%%%%%%%%%%%%%%%%%%%%%%%%%%%%%%%%%%%%%%%%%
%%%%%%%%%%%%%%%%%%%%%%%%%%%%%%%%%%%%%%%%%%%%%%%%%%%%%%%%%%%%%%%%%%%%%%%%%%%%%%%
%%%%%%%%%%%%%%%%%%%%%%%%%%%%%%%%%%%%%%%%%%%%%%%%%%%%%%%%%%%%%%%%%%%%%%%%%%%%%%%
\section{Основы}

%%%%%%%%%%%%%%%%%%%%%%%%%%%%%%%%%%%%%%%%%%%%%%%%%%%%%%%%%%%%%%%%%%%%%%%%%%%%%%%
%%%%%%%%%%%%%%%%%%%%%%%%%%%%%%%%%%%%%%%%%%%%%%%%%%%%%%%%%%%%%%%%%%%%%%%%%%%%%%%
%%%%%%%%%%%%%%%%%%%%%%%%%%%%%%%%%%%%%%%%%%%%%%%%%%%%%%%%%%%%%%%%%%%%%%%%%%%%%%%
\subsection{Первые шаги}

\begin{frame}[fragile]{\insertsubsection}
\vspace{-2ex}
\begin{itemize}
  \item Минимальный документ \LaTeX{} (на русском языке):

\inputcode{basics.tex}
\item Команды начинаются с \emph{обратного слэша} \keystrokebftt{\bs}.
\item Каждый документ начинается с команды \cmdbs{documentclass}.
\item \emph{Аргумент} в фигурных скобках \keystrokebftt{\{} \keystrokebftt{\}}
  указывает \LaTeX'у что именно за документ мы пишем: \bftt{article} означает статья.
\item Значок процента \keystrokebftt{\%} начинает \emph{комментарий} --- \LaTeX{}
проигнорирует оставшуюся часть строки.
\item Пока что переносы для русского языка отключены, мы рассмотрим их во
  второй части курса
\end{itemize}
\end{frame}

%%%%%%%%%%%%%%%%%%%%%%%%%%%%%%%%%%%%%%%%%%%%%%%%%%%%%%%%%%%%%%%%%%%%%%%%%%%%%%%
%%%%%%%%%%%%%%%%%%%%%%%%%%%%%%%%%%%%%%%%%%%%%%%%%%%%%%%%%%%%%%%%%%%%%%%%%%%%%%%
%%%%%%%%%%%%%%%%%%%%%%%%%%%%%%%%%%%%%%%%%%%%%%%%%%%%%%%%%%%%%%%%%%%%%%%%%%%%%%%
\begin{frame}[fragile]{\insertsubsection{} в \wllogo}
\begin{itemize}
  \item Overleaf --- это вебсайт, позволяющий набирать документы \LaTeX{} в интернете.
\item Он <<компилирует>> ваш текст \LaTeX{} автоматически и показывает результат.
\vskip 2em
\begin{center}
\begin{printout}
\href{\wlnewdoc{basics.tex}}{%
Щёлкните здесь, чтобы открыть пример в \wllogo{}}
\end{printout}
\vspace{1ex}\scriptsize
Для лучшего результата воспользуйтесь браузером
\href{http://www.google.com/chrome}{Google Chrome} или новой версией браузера
\href{http://www.mozilla.org/en-US/firefox/new/}{FireFox}.
\end{center}
\vspace{1ex}
\item В дальнейшем при просмотре сладов пробуйте набирать рассматриваемые
  примеры в Overleaf.
\item \textbf{Нет, правда, вы должны их попробовать по мере просмотра!}
\end{itemize}
\end{frame}

%%%%%%%%%%%%%%%%%%%%%%%%%%%%%%%%%%%%%%%%%%%%%%%%%%%%%%%%%%%%%%%%%%%%%%%%%%%%%%%
%%%%%%%%%%%%%%%%%%%%%%%%%%%%%%%%%%%%%%%%%%%%%%%%%%%%%%%%%%%%%%%%%%%%%%%%%%%%%%%
%%%%%%%%%%%%%%%%%%%%%%%%%%%%%%%%%%%%%%%%%%%%%%%%%%%%%%%%%%%%%%%%%%%%%%%%%%%%%%%
\subsection{Набор текста}
\begin{frame}[fragile]{\insertsubsection{}}
\small
\begin{itemize}
\item Набирайте ваш текст между \cmdbegin{document} и \cmdend{document}.
\item По большей части вы можете просто вводить текст как есть.

\begin{exampletwouptiny}
Слова разделяются одним
пробелом или несколькими.

Абзацы разделяются одной
пустой строкой или
несколькими.
\end{exampletwouptiny}
\item Несколько пробелов подряд в исходном тексте воспринимаются как один.

\begin{exampletwouptiny}
Шла     Саша    по  шоссе
и      сосала      сушку.
\end{exampletwouptiny}
\end{itemize}
\end{frame}

%%%%%%%%%%%%%%%%%%%%%%%%%%%%%%%%%%%%%%%%%%%%%%%%%%%%%%%%%%%%%%%%%%%%%%%%%%%%%%%
%%%%%%%%%%%%%%%%%%%%%%%%%%%%%%%%%%%%%%%%%%%%%%%%%%%%%%%%%%%%%%%%%%%%%%%%%%%%%%%
%%%%%%%%%%%%%%%%%%%%%%%%%%%%%%%%%%%%%%%%%%%%%%%%%%%%%%%%%%%%%%%%%%%%%%%%%%%%%%%
\begin{frame}[fragile]{\insertsubsection{}: особенности}
\vspace{-3ex}
\small
\begin{itemize}
\item Символы кавычек вводятся хитро: используются символы \keystroke{<} и
  \keystroke{>} для <<ёлочек>>, и запятая \keystroke{,} с обратной кавычкой
  \keystroke{\char96} для ,,лапок``.
\begin{exampletwouptiny}
Кавычки-ёлочки: <<текст>>.

Кавычки-лапки: ,,текст``.
\end{exampletwouptiny}
\item Некоторые часто встречающиеся символы имеют специальный смысл в \LaTeX:

\begin{tabular}{cl}
\keystrokebftt{\%} & значок процента     \\
\keystrokebftt{\#} & значок диеза / хэша \\
\keystrokebftt{\&} & амперсанд           \\
\keystrokebftt{\$} & значок доллара      \\
\end{tabular}
\item Если ввести эти символы непосредственно, то получится ошибка. Если
  нужно, чтобы эти значки появились в документе, то их придётся \emph{экранировать}
  с помощью обратного слэша.
\begin{exampletwoup}
\$\%\&\#!
\end{exampletwoup}
\end{itemize}
\end{frame}

%%%%%%%%%%%%%%%%%%%%%%%%%%%%%%%%%%%%%%%%%%%%%%%%%%%%%%%%%%%%%%%%%%%%%%%%%%%%%%%
%%%%%%%%%%%%%%%%%%%%%%%%%%%%%%%%%%%%%%%%%%%%%%%%%%%%%%%%%%%%%%%%%%%%%%%%%%%%%%%
%%%%%%%%%%%%%%%%%%%%%%%%%%%%%%%%%%%%%%%%%%%%%%%%%%%%%%%%%%%%%%%%%%%%%%%%%%%%%%%
\begin{frame}[fragile]{\insertsubsection{}: особенности}
\vspace{-2ex}
\small
Следует также отметить, что в печатной продукции встречаются три
вида <<тире>>. В \LaTeX{} они вводятся специальным образом:
\vspace{-1ex}
\begin{description}
  \item[\keystrokebftt{-}] дефис
  \item[\keystrokebftt{-}\keystrokebftt{-}] короткое тире, используется
    при перечислении или задании дапазонов
  \item[\keystrokebftt{-}\keystrokebftt{-}\keystrokebftt{-}] длинное тире,
    используется в прямой речи или как собственно тире в тексте
  \item[\keystrokebftt{\$}\keystrokebftt{-}\keystrokebftt{\$}] минус в
    математическом режиме, $x - y$
\end{description}
\begin{exampletwouptiny}
Бело-голубые цвета.

Теорема Гаусса--Маркова, с.~2--7.

Единица --- ноль!

$2 - 1 = 0$.
\end{exampletwouptiny}
\end{frame}

%%%%%%%%%%%%%%%%%%%%%%%%%%%%%%%%%%%%%%%%%%%%%%%%%%%%%%%%%%%%%%%%%%%%%%%%%%%%%%%
%%%%%%%%%%%%%%%%%%%%%%%%%%%%%%%%%%%%%%%%%%%%%%%%%%%%%%%%%%%%%%%%%%%%%%%%%%%%%%%
%%%%%%%%%%%%%%%%%%%%%%%%%%%%%%%%%%%%%%%%%%%%%%%%%%%%%%%%%%%%%%%%%%%%%%%%%%%%%%%
\begin{frame}[fragile]{Handling Errors}
\begin{itemize}
\item \LaTeX{} can get confused when it is trying to compile your document. If
it does, it stops with an error, which you must fix before it will produce
any output.
\item For example, if you misspell \cmdbs{emph} as \cmdbs{meph}, \LaTeX{} will
stop with an ``undefined control sequence'' error, because ``meph'' is not
one of the commands it knows.
\end{itemize}
\begin{block}{Advice on Errors}
\begin{enumerate}
\item Don't panic! Errors happen.
\item Fix them as soon as they arise --- if what you just typed caused an error,
you can start your debugging there.
\item If there are multiple errors, start with the first one --- the cause may
even be above it.
\end{enumerate}
\end{block}
\end{frame}

%%%%%%%%%%%%%%%%%%%%%%%%%%%%%%%%%%%%%%%%%%%%%%%%%%%%%%%%%%%%%%%%%%%%%%%%%%%%%%%
%%%%%%%%%%%%%%%%%%%%%%%%%%%%%%%%%%%%%%%%%%%%%%%%%%%%%%%%%%%%%%%%%%%%%%%%%%%%%%%
%%%%%%%%%%%%%%%%%%%%%%%%%%%%%%%%%%%%%%%%%%%%%%%%%%%%%%%%%%%%%%%%%%%%%%%%%%%%%%%
\begin{frame}[fragile]{Упражнение наборщика 1}

\begin{tblock}{Наберите этот текст в \LaTeX:\footnotemark}
По итогам 2015 года ВВП России снизился (на 2.8\%), впервые после кризиса
2008--2009 годов. Инфляция выросла до 12.9\%. Реальные доходы населения
снизились на 3.2\%. В то же время произошло снижение оттока капитала почти в
3 раза (до \$58 млрд).
\end{tblock}
\footnotetext{\href{https://ru.wikipedia.org/wiki/\%D0\%AD\%D0\%BA\%D0\%BE\%D0\%BD\%D0\%BE\%D0\%BC\%D0\%B8\%D0\%BA\%D0\%B0_\%D0\%A0\%D0\%BE\%D1\%81\%D1\%81\%D0\%B8\%D0\%B8}{https://ru.wikipedia.org/wiki/Экономика\_{}России}}
\vskip 1ex
\begin{printout}\href{\wlnewdoc{basics-exercise-1.tex}}{%
Щёлкните, чтобы открыть это упражнение в \wllogo{}}
\end{printout}
\begin{itemize}
\item Указание: обращайте внимание на специальные символы!
\item Когда попробуете,
\ovllink{\href{\wlnewdoc{basics-exercise-1-solution.tex}}{%
щёлкните, чтобы посмотреть решение}}.
\vspace{2pt}
\end{itemize}
\end{frame}

%%%%%%%%%%%%%%%%%%%%%%%%%%%%%%%%%%%%%%%%%%%%%%%%%%%%%%%%%%%%%%%%%%%%%%%%%%%%%%%
%%%%%%%%%%%%%%%%%%%%%%%%%%%%%%%%%%%%%%%%%%%%%%%%%%%%%%%%%%%%%%%%%%%%%%%%%%%%%%%
%%%%%%%%%%%%%%%%%%%%%%%%%%%%%%%%%%%%%%%%%%%%%%%%%%%%%%%%%%%%%%%%%%%%%%%%%%%%%%%
\subsection{Набор математики}
\begin{frame}[fragile]{\insertsubsection{}: Dollar Signs}
\begin{itemize}
\item Why are dollar signs \keystrokebftt{\$} special? We use them to mark mathematics in text.\\[1ex]
\begin{exampletwouptiny}
% not so good:
Let a and b be distinct positive
integers, and let c = a - b + 1.

% much better:
Let $a$ and $b$ be distinct positive
integers, and let $c = a - b + 1$.
\end{exampletwouptiny}
\item Always use dollar signs in pairs --- one to begin the mathematics, and one
to end it.
\item \LaTeX{} handles spacing automatically; it ignores your spaces.
\begin{exampletwouptiny}
Let $y=mx+b$ be \ldots

Let $y = m x + b$ be \ldots
\end{exampletwouptiny}
\end{itemize}
\end{frame}

%%%%%%%%%%%%%%%%%%%%%%%%%%%%%%%%%%%%%%%%%%%%%%%%%%%%%%%%%%%%%%%%%%%%%%%%%%%%%%%
%%%%%%%%%%%%%%%%%%%%%%%%%%%%%%%%%%%%%%%%%%%%%%%%%%%%%%%%%%%%%%%%%%%%%%%%%%%%%%%
%%%%%%%%%%%%%%%%%%%%%%%%%%%%%%%%%%%%%%%%%%%%%%%%%%%%%%%%%%%%%%%%%%%%%%%%%%%%%%%
\begin{frame}[fragile]{\insertsubsection{}: Обозначения}
\vspace{-3ex}
\begin{itemize}
\item Используйте <<крышку>> \keystrokebftt{\char94} для верхних индексов и
  подчёркивание \keystrokebftt{\_} для нижних.

\begin{exampletwouptiny}
$y = c_2 x^2 + c_1 x + c_0$
\end{exampletwouptiny}

\item Используйте фигурные скобки \keystrokebftt{\{} \keystrokebftt{\}} для
группировки верхних и нижних индексов.

\begin{exampletwouptiny}
$F_n = F_n-1 + F_n-2$ % ой!

$F_n = F_{n-1} + F_{n-2}$
\end{exampletwouptiny}

\item Греческие буквы и математические операции набираются с помощью команд.

\begin{exampletwouptiny}
$\mu = A e^{Q/RT}$

$\Omega =
  \sum_{k=1}^{n} \omega_k$
\end{exampletwouptiny}
\end{itemize}
\end{frame}

%%%%%%%%%%%%%%%%%%%%%%%%%%%%%%%%%%%%%%%%%%%%%%%%%%%%%%%%%%%%%%%%%%%%%%%%%%%%%%%
%%%%%%%%%%%%%%%%%%%%%%%%%%%%%%%%%%%%%%%%%%%%%%%%%%%%%%%%%%%%%%%%%%%%%%%%%%%%%%%
%%%%%%%%%%%%%%%%%%%%%%%%%%%%%%%%%%%%%%%%%%%%%%%%%%%%%%%%%%%%%%%%%%%%%%%%%%%%%%%
\begin{frame}[fragile]{\insertsubsection{}: Выключные уравнения}
\begin{itemize}
  \item Если уравнение большое и страшное, \emph{вынесите} его на отдельную
    строку с помощью
\cmdbegin{equation} и \cmdend{equation}.
\vspace{1ex}
\begin{exampletwouptiny}
Корни квадратного уравнения
можно найти по формуле
\begin{equation}
x =
\frac{-b \pm\sqrt{b^2 - 4ac}}
     {2a}
\end{equation}
где $a$, $b$ и $c$ --- \dots
\end{exampletwouptiny}

{\scriptsize Замечание: \LaTeX{} почти всегда игнорирует пробелы при наборе
  математики, но пустые строки он воспринимает как ошибки --- не вставляйте
  пустые строки в формулы.}
\end{itemize}
\end{frame}

%%%%%%%%%%%%%%%%%%%%%%%%%%%%%%%%%%%%%%%%%%%%%%%%%%%%%%%%%%%%%%%%%%%%%%%%%%%%%%%
%%%%%%%%%%%%%%%%%%%%%%%%%%%%%%%%%%%%%%%%%%%%%%%%%%%%%%%%%%%%%%%%%%%%%%%%%%%%%%%
%%%%%%%%%%%%%%%%%%%%%%%%%%%%%%%%%%%%%%%%%%%%%%%%%%%%%%%%%%%%%%%%%%%%%%%%%%%%%%%
\begin{frame}[fragile]{Отступление: Окружения}
\begin{itemize}
\item \bftt{equation} is an \emph{environment} --- a context.
\item A command can produce different output in different contexts.
\begin{exampletwouptiny}
We can write
$ \Omega = \sum_{k=1}^{n} \omega_k $
in text, or we can write
\begin{equation}
  \Omega = \sum_{k=1}^{n} \omega_k
\end{equation}
to display it.
\end{exampletwouptiny}
\vskip 2ex
\item Note how the $\Sigma$ is bigger in the \bftt{equation} environment, and
how the subscripts and superscripts change position, even though we used the
same commands.
\vskip 1em
{\scriptsize In fact, we could have written \bftt{\$...\$} as
\cmdbegin{math}\bftt{...}\cmdend{math}.}
\end{itemize}
\end{frame}

%%%%%%%%%%%%%%%%%%%%%%%%%%%%%%%%%%%%%%%%%%%%%%%%%%%%%%%%%%%%%%%%%%%%%%%%%%%%%%%
%%%%%%%%%%%%%%%%%%%%%%%%%%%%%%%%%%%%%%%%%%%%%%%%%%%%%%%%%%%%%%%%%%%%%%%%%%%%%%%
%%%%%%%%%%%%%%%%%%%%%%%%%%%%%%%%%%%%%%%%%%%%%%%%%%%%%%%%%%%%%%%%%%%%%%%%%%%%%%%
\begin{frame}[fragile]{Отступление: Окружения}
\begin{itemize}
\item Команды \cmdbs{begin} и \cmdbs{end} используются для создания разных
  окружений.

\item Окружения \bftt{itemize} и \bftt{enumerate} печатают списки.
\vskip 1ex

\begin{exampletwouptiny}
% ненумерованный список
\begin{itemize}
\item Печенье
\item Чай
\end{itemize}

%нумерованный список
\begin{enumerate}
\item Печенье
\item Чай
\end{enumerate}
\end{exampletwouptiny}
\end{itemize}
\end{frame}

%%%%%%%%%%%%%%%%%%%%%%%%%%%%%%%%%%%%%%%%%%%%%%%%%%%%%%%%%%%%%%%%%%%%%%%%%%%%%%%
%%%%%%%%%%%%%%%%%%%%%%%%%%%%%%%%%%%%%%%%%%%%%%%%%%%%%%%%%%%%%%%%%%%%%%%%%%%%%%%
%%%%%%%%%%%%%%%%%%%%%%%%%%%%%%%%%%%%%%%%%%%%%%%%%%%%%%%%%%%%%%%%%%%%%%%%%%%%%%%
\begin{frame}[fragile]{Отступление: Пакеты}
\vspace{-3ex}
\begin{itemize}
  \item Все команды и окружения, которые мы до сих пор
    использовали,\footnote{Кроме \cmdbs{setmainfont} при подключении шрифта
    с русскими буквами} были встроены в \LaTeX.

\item \emph{Пакеты} это библиотеки дополнительных команд и окружений.
  Существуют тысячи пакетов в свободном доступе.

\item Чтобы загрузить пакет и пользоваться реализованными в нём командами,
  нужно использовать команду \cmdbs{usepackage} в \emph{преамбуле}.

\item Пример: \bftt{amsmath} от Американского математического общества.
\begin{code}
\documentclass{article}
\usepackage{amsmath} % преамбула
\begin{document}
% теперь можно использовать команды из amsmath...
\end{document}
\end{code}
\vspace*{2pt}
\end{itemize}
\end{frame}

%%%%%%%%%%%%%%%%%%%%%%%%%%%%%%%%%%%%%%%%%%%%%%%%%%%%%%%%%%%%%%%%%%%%%%%%%%%%%%%
%%%%%%%%%%%%%%%%%%%%%%%%%%%%%%%%%%%%%%%%%%%%%%%%%%%%%%%%%%%%%%%%%%%%%%%%%%%%%%%
%%%%%%%%%%%%%%%%%%%%%%%%%%%%%%%%%%%%%%%%%%%%%%%%%%%%%%%%%%%%%%%%%%%%%%%%%%%%%%%
\begin{frame}[fragile]{\insertsubsection{}: Примеры с \bftt{amsmath}}
\vspace{-2ex}
\begin{itemize}
\item Можно использовать \bftt{equation*} (<<equation со звёздочкой>>) для
  уравнений без номера.
\begin{exampletwouptiny}
\begin{equation*}
\Omega=\sum_{k=1}^{n}\omega_k
\end{equation*}
\end{exampletwouptiny}
\item \LaTeX{} трактует рядом стоящие буквы как отдельные переменные, и это
  не всегда то, что хочется. В \bftt{amsmath} определены команды для многих
  часто используемых математических операторов.
\begin{exampletwouptiny}
\begin{equation*} % плохо!
 min_{x,y}(1-x)^2+10(y-x^2)^2
\end{equation*}
\begin{equation*} % хорошо!
\min_{x,y}(1-x)^2+10(y-x^2)^2
\end{equation*}
\end{exampletwouptiny}
\end{itemize}
\end{frame}

%%%%%%%%%%%%%%%%%%%%%%%%%%%%%%%%%%%%%%%%%%%%%%%%%%%%%%%%%%%%%%%%%%%%%%%%%%%%%%%
%%%%%%%%%%%%%%%%%%%%%%%%%%%%%%%%%%%%%%%%%%%%%%%%%%%%%%%%%%%%%%%%%%%%%%%%%%%%%%%
%%%%%%%%%%%%%%%%%%%%%%%%%%%%%%%%%%%%%%%%%%%%%%%%%%%%%%%%%%%%%%%%%%%%%%%%%%%%%%%
\begin{frame}[fragile]{\insertsubsection{}: Примеры с \bftt{amsmath}}
\vspace{-2ex}
\begin{itemize}
\item Если подходящего оператора не нашлось, можно воспользоваться \cmdbs{operatorname}.
\vspace{1ex}
\begin{exampletwouptiny}
\begin{equation*}
\beta_i =
\frac{\operatorname{Cov}(R_i,R_m)}
     {\operatorname{Var}(R_m)}
\end{equation*}
\end{exampletwouptiny}
\end{itemize}
\end{frame}

%%%%%%%%%%%%%%%%%%%%%%%%%%%%%%%%%%%%%%%%%%%%%%%%%%%%%%%%%%%%%%%%%%%%%%%%%%%%%%%
%%%%%%%%%%%%%%%%%%%%%%%%%%%%%%%%%%%%%%%%%%%%%%%%%%%%%%%%%%%%%%%%%%%%%%%%%%%%%%%
%%%%%%%%%%%%%%%%%%%%%%%%%%%%%%%%%%%%%%%%%%%%%%%%%%%%%%%%%%%%%%%%%%%%%%%%%%%%%%%
\begin{frame}[fragile]{\insertsubsection{}: Примеры с \bftt{amsmath}}
\vspace{-2ex}
\begin{itemize}{\small
\item Можно выравнивать несколько уравнений по знакам равенства
\begin{printout}
\begin{align*}
(x+1)^3 &= (x+1)(x+1)(x+1) \\
        &= (x+1)(x^2 + 2x + 1) \\
        &= x^3 + 3x^2 + 3x + 1
\end{align*}
\end{printout}
с помощью окружения \bftt{align*}.

% for whatever reason, this doesn't play well with the twoup environment
\begin{code}
\begin{align*}
(x+1)^3 &= (x+1)(x+1)(x+1) \\
        &= (x+1)(x^2 + 2x + 1) \\
        &= x^3 + 3x^2 + 3x + 1
\end{align*}
\end{code}
\item Символ амперсанда \keystrokebftt{\&} отделяет левую колонку (до знака
  $=$) от правой колонки (после знака $=$).
\item Двойной обратный слэш \keystrokebftt{\bs}\keystrokebftt{\bs} начинает
  новую строку.
}\end{itemize}
\end{frame}


%%%%%%%%%%%%%%%%%%%%%%%%%%%%%%%%%%%%%%%%%%%%%%%%%%%%%%%%%%%%%%%%%%%%%%%%%%%%%%%
%%%%%%%%%%%%%%%%%%%%%%%%%%%%%%%%%%%%%%%%%%%%%%%%%%%%%%%%%%%%%%%%%%%%%%%%%%%%%%%
%%%%%%%%%%%%%%%%%%%%%%%%%%%%%%%%%%%%%%%%%%%%%%%%%%%%%%%%%%%%%%%%%%%%%%%%%%%%%%%
\begin{frame}[fragile]{Упражнение наборщика 2}
\vspace{-1.5ex}
\abovedisplayskip=1pt
\belowdisplayskip=1pt
\begin{tblock}{Наберите этот текст в \LaTeX:}
Пусть $X_1, X_2, \ldots, X_n$ --- последовательность независимых одинаково
распределённых случайных величин, для которых
$\operatorname{E}[X_i] = \mu$ и
$\operatorname{Var}[X_i] = \sigma^2 < \infty$, и пусть
\begin{equation*}
S_n = \frac{1}{n}\sum_{i}^{n} X_i
\end{equation*}
обозначает их среднее арифметическое. Тогда при $n$, стремящемся к
бесконечности, последовательность случайных величин
$\sqrt{n}(S_n - \mu)$ сходится по распределению к нормальной $N(0, \sigma^2)$
величине.
\end{tblock}
\vspace{-0.5ex}
\begin{printout}\href{\wlnewdoc{basics-exercise-2.tex}}{%
Щёлкните, чтобы открыть это упражнение в \wllogo{}}
\end{printout}
\vspace{-2ex}
\begin{itemize}
\item Указание: Символ $\infty$ набирается как \cmdbs{infty}.
\item Когда попробуете,
\ovllink{\href{\wlnewdoc{basics-exercise-2-solution.tex}}{%
щёлкните, чтобы увидеть решение}}.
\end{itemize}
\end{frame}

%%%%%%%%%%%%%%%%%%%%%%%%%%%%%%%%%%%%%%%%%%%%%%%%%%%%%%%%%%%%%%%%%%%%%%%%%%%%%%%
%%%%%%%%%%%%%%%%%%%%%%%%%%%%%%%%%%%%%%%%%%%%%%%%%%%%%%%%%%%%%%%%%%%%%%%%%%%%%%%
%%%%%%%%%%%%%%%%%%%%%%%%%%%%%%%%%%%%%%%%%%%%%%%%%%%%%%%%%%%%%%%%%%%%%%%%%%%%%%%
\begin{frame}{Конец первой части}
\begin{itemize}
\item Поздравляем! Вы уже узнали, как \ldots
\begin{itemize}
\item Набирать текст в \LaTeX.
\item Использовать множество разных команд.
\item Исправлят ошибки, когда они возникают.
\item Красиво набирать математические формулы.
\item Пользоваться несколькими различными окружениями.
\item Подключать пакеты.
\end{itemize}
\item Это здорово!
\item Во второй части вы увидите, как, используя \LaTeX{}, набирать
  структурированные документы с разделами, перекрёстными ссылками,
  рисунками, таблицами и списком литературы. До встречи!
\end{itemize}
\end{frame}

\end{document}
