\documentclass{article}
\usepackage{fontspec}
\setmainfont{CMU Serif}
\usepackage[russian]{babel}
\usepackage[width=6cm,height=10cm,left=0pt,top=0pt]{geometry}

\usepackage[colorinlistoftodos]{todonotes}
\makeatletter
\@todonotes@SetTodoListName{Список что сделать}%
\@todonotes@SetMissingFigureText{Рисунок}%
\@todonotes@SetMissingFigureUp{Отсутствует}%
\@todonotes@SetMissingFigureDown{рисунок}%
\makeatother

\begin{document}
\listoftodos
\section{Самосинхронизация}
Самосинхронизацию в
\todo{a как обстоят дела с UTF-16?}
UTF-8 можно рассмотреть когда вашей программе
подаются случайные байты и вам нужно определить начало первого
символа. Первичным признаком является сброшенный старший бит байта
--- это ASCII-символ. Если же он установлен, то пропускаем те
байты, у которых сброшен бит перед
\todo[color=green!50]{что если теряется несколько битов?}
старшим.
В остальных случаях
можно продолжать посимвольное поточное раскодирование.
\end{document}
