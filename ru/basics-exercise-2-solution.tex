\documentclass{article}
\usepackage{amsmath}
\usepackage{fontspec}
\setmainfont{CMU Serif}
\begin{document}

Пусть $X_1, X_2, \dots, X_n$ --- последовательность независимых
одинаково распределённых случайных величин, для которых
$\operatorname{E}[X_i] = \mu$ и
$\operatorname{Var}[X_i] = \sigma^2 < \infty$, и пусть
\begin{equation*}
S_n = \frac{1}{n}\sum_{i=1}^{n} X_i
\end{equation*}
обозначает их среднее арифметическое. Тогда при $n$,
стремящемся к бесконечности, последовательность случайных
величин $\sqrt{n}(S_n - \mu)$ сходится по распределению
к нормальной $N(0, \sigma^2)$ величине.

% бонусные очки: нормальное распределение N обычно обозначается
% каллиграфическим шрифтом; этого можно добиться вот таким
% способом: $\mathcal{N}(0, \sigma^2)$.

\end{document}

